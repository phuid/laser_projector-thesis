\chapter{Použité technologie, techniky, pojmy}
nodejs
ZeroMQ

\section{SPI}\label{sec:spi}
SPI (anglicky Serial Peripheral Interface) je komunikační rozhraní používané k přenosu dat mezi ovládajícím zařízením (často mikrokontrolerem) a jedním, nebo více periferními integrovynými obvody (periferiemi).
Používá separátní kontakty (linky) pro hodinový signál, kterým mikrokontroler ovládá rychlost přenosu dat, a datové linky.
Datové linky jsou fixně využívané k posílání dat buď periferií -- linka POCI (Peripheral Out, Controller In), nebo mikrokontrolerem -- linka PICO (Peripheral In, Controller Out).
Dále používá separátní kontakt(y) pro výběr periferie, se kterou mikrokontroler komunikuje. Tomuto kontaktu se říká chip select (zkráceně CS). Pro každou periferii je potřeba připojit separátní CS kontakt.~\cite{sparkfun-spi}

\section{I$^{2}$C}
I$^{2}$C (anglicky Inter-Integrated Circuit) je komunikační protokol umožňující komunikaci mezi jednou nebo více periferiemi a jedním, nebo více mikrokontrolery.
Podobně jako SPI je úrčen ke komunikaci na krátké vzdálenosti v jednom zařízení.
Narozdíl od SPI využívá pouze dva kontakty k výměně informací včetně výběru připojené periferie, se kterou mikrokontroler chce komunikovat.
Zároveň umožňuje fungování systémů s několika kontrolery.~\cite{sparkfun-i2c}

Dva kontakty při použití tohoto protokolu stačí i na výběr zařízení, se kterým chce mikrokontroler komunikovat.
Po začátku komunikace totiž sběrnicí\footnote{Sběrnice I$^{2}$C je systém zařízení propojený protokolem I$^{2}$C na dvou stejných kontaktech.} mikrokontroler pošle tzv. adresu zařízení se kterým chce komunikovat~\cite{sparkfun-i2c}.
Adresa zařízení je sestavena ze sedmi bitů a může tedy dosahovat 127 různých hodnot (kromě systémů s 10bitovou adresou; Ty jsou ale vzácné).
Většinou je pro danou periferii nastavena od výroby, ui některých je ale možné ji softwarově změnit.

\section{Pull-up a pull-down rezistory}
Pokud při čtení z GPIO pinů mikrokontrolerů či jiných zařízení k danému pinu není nic připojeno, je těžké předpovědět, jaký stav na tomto pinu přečteme.
Když tento fenomén nastane, říká se, že je pin floating (plovoucí)~\cite{sparkfun-pud}.
Může nastat například, chceme-li číst stav tlačítka, přes které by byl pin připojen na zem. Když tlačítko není zmáčknuté, pin na zem připojen není a tudíž je floating.

V takových případech se k pinu připojuje tzv. pull-up rezistor, který zaručí, že z pinu v případě, že tlačítko stisknuté není, přečteme hodnotu HIGH.
Rozdíl mezi pull-up a pull-down rezistory je pouze v tom, že pull-up rezistory připojují pin k napětí, na kterém zařízení pracuje (často 3,3 nebo 5~V) a pull-down rezistory rezistory připojují pin k zemi.~\cite{sparkfun-pud}


obdelnikovy prubeh
pwm
power delivery
github? - cuz it was vystaveno there and will continue to develop

hostapd dnsmasq dhcpcd systemd

čislovani GPIO
interrupt