\chapter{Laser safety}
\fxnote{citovat z~\url{https://dspace.vut.cz/server/api/core/bitstreams/a55c1118-9166-4449-806a-c2a73bfeff66/content}}

3.3 Laserová bezpečnost [7,8]
Při práci s~lasery vzniká možnost ohrožení zdraví laserovým zářením. Podle nařízení vlády
č. 1/2008 Sb. O~ochraně zdraví před neionizujícím zářením, ve~znění pozdějších předpisů se
optickým zářením se ~pro~účely tohoto nařízení rozumí záření z~umělých zdrojů odpovídající
vlnovým délkám od~100 nm ~do~1 mm, jehož spektrum se~dělí na:
• ultrafialové (100 nm~až 400 nm)
• viditelné záření (380 nm~až 780 nm)
• infračervené (780 nm~až 1 mm)
Laserem se~rozumí jakékoliv zařízení, které je~určeno k~vytváření nebo zesilování
elektromagnetického záření primárně procesem kontrolované stimulované emise. Laserová
záření jsou elektromagnetické vlny stejné fyzikální podstaty jako vlny vznikající v~přírodě,
ovšem jejich intenzita a~rovnoběžnost svazku je~podstatně vyšší. V~případě vystavení člověka
působení laserového záření vzniká nebezpečí újmy na~zdraví, postihující nejvíce oči a~kůži.
Ve většině případů se~nedá před poškozením oka~ochránit odkloněním hlavy nebo
zavřením očních víček, jelikož takováto reakce je~příliš pomalá. Pro~ochranu zdraví jsou tudíž
19
stanovena pravidla, která zaručují, že k~poškození zdraví nedojde. Maximální přípustná dávka
ozáření (MPE) je~hodnota laserového záření, kdy~při vystavení lidské pokožky nebo oka
nedojde k~okamžitému nebo pozdějšímu poranění. Hodnota MPE~je~závislá na~vlnové délce,
době ozáření, typu ozařované tkáně a~při ozáření oka~na~velikosti obrazu na~sítnici.
V Tabulka 1 jsou upraveny nejvyšší přípustné hodnoty expozice pro~přímý pohled do
laserového svazku nebo přímý pohled do~zrcadlově odraženého svazku.
Tabulka 1 Nejvyšší přípustná hodnota expozice při přímém působení laserového záření na~rohovku oka
(přímý pohled do~svazku) [7]
V Tabulka 2 jsou upraveny nejvyšší přípustné hodnoty ozáření rohovky oka~při sledování
plošného laserového zdroje nebo laserového svazku po~difúzním odrazu.
20
Tabulka 2 Nejvyšší přípustné hodnoty ozáření rohovky oka~při pozorování plošného laserového zdroje
nebo laserového svazku po~difúzním odrazu [7]
V Tabulka 3 jsou zobrazeny hodnoty nejvyššího přípustného ozáření při expozici
laserového záření na~kůži.
Tabulka 3 Nejvyšší přípustné ozáření při expozici laserového záření na~kůži [7]
Tabulka 4 Hodnoty konstant pro~tabulky výše [7]
21
Korekční faktory použité v~tabulkách 1 – 3 jsou vyjádřeny v~Tabulka 4.
3.4 Třídy laserů
• 1 – zařízení třídy 1 jsou bezpečná včetně dlouhodobého pozorování světelného
svazku nebo jeho sledování pomocí čoček či dalekohledů. To~této kategorie spadají
i vysokovýkonné lasery, které jsou zcela zakrytovány a~neumožní průnik paprsku
do okolí, při případném otevření se~celé zařízení vypne.
• 1M – zařízení spadající do~třídy 1M jsou bezpečná i~při dlouhodobém sledování
světelného svazku, nejsou ovšem bezpečná při sledování pomocí optických
pomůcek (čočky, dalekohledy).
• 2 – Jedná se~lasery v~rozsahu vlnových délek 400 – 700 nm. Přímý dlouhodobý
pohled do~světelného svazku může způsobit poškození zraku. Reakční doba lidí při
oslnění (zavření víček, odvrácení hlavy) se~pohybuje okolo 0,25s. Pokud člověk při
oslnění do~této doby zareaguje, neměl by~mít trvale poškozený zrak, může ovšem
dojít ke~krátkodobým poruchám vidění, které v~pracovních provozech významně
ovlivňují bezpečnost práce.
• 2M – stejné jako třída 2 s~tím, že při použití optických pomůcek není garantováno
po vystavení lidského oka~záření po~dobu do~0,25s, že nedojde k~nevratnému
poškození zraku.
• 3R - vyzařované záření může při přímém sledování svazku překročit maximální
přípustnou dávku ozáření, nebezpečí poškození zraku je~ovšem relativně nízké,
jelikož limity třídy 3R jsou pouze pětinásobkem třídy 2. Zařízení s~laserem třídy
3R by~měla být používána pouze tam, kde~je~pohled do~svazku nepravděpodobný.
• 3B – při pohledu do~světelného svazku je~vážné riziko trvalého poškození zraku,
včetně náhodných a~krátkodobých ozáření. Záření z~difuzního odrazu jsou běžně
bezpečná. Může dojít k~malým poraněním pokožky, laser může být též příčinnou
vzniku požáru.
• 4 – Pohled do~laseru spadajícího do~třídy 4 je~značně nebezpečný, ozáření pokožky
představuje taktéž velké nebezpečí. Nebezpečí mohou představovat i~odražené a
rozptýlené paprsky. Laser představuje i~riziko vzniku požáru.
Podle [7] je~limit pro~třídu 3A při působení záření na~lidské oko~po~dobu nad~0,25s 5mW a
25W/m2, limity pro~třídu 3B je~0,5W v~rozsahu vlnových délek 400 – 700nm.
22
Lasery použité v~práci (červený 650nm, 100mW – zelený 532nm, 100mW – modrý 450
nm, 150mW) jsou tedy třídy 3B a~přímý pohled do~světelného svazku je~nebezpečný, světlo z
obrazců vykreslovaných na~stěnu a~dopadající na~sítnici oka~vlivem difuzního odrazu není pro
zrak hrozbou.