\section{Pouzdro}

V~rámci popularizace technologie se~může hodit projektor předvádět na~různých místech. Proto bude potřeba,  aby~byl projektor přenosný a~při přemisťování se~nerozbil.

Pouzdro je~tedy navrženo tak,  aby~bylo odolné proti nárazům a~zachovalo všechny součásti v~bezpečí. Každá součástka má své místo,  kde~je~držena ze~všech stran.  Pro~výrobu pouzdra by~bylo vhodné využít technologii 3D tisku. Ta~umožňuje tvorbu komplexních geometrických tvarů přímo  pro~potřeby konkrétního modelu a~zároveň umožňuje snadnou iteraci a~úpravu designu pouzdra při nalezení chyb.

Pouzdro bylo navrženo  v~programu Autodesk Fusion.

\subsection{Priority designu} \label{sec:krabick-design-priorities}
\subsubsection{Chlazení}
Největší část projektoru je~hliníkový chladič s~větrákem. Už od~začátku práce  byl~tento chladič vybrán,  aby~byl připevněn k~řídící desce galvanometrů. Problematika jejího zahřívání je~popsaná v~kapitole~\ref{sec:galvoboard-chips-heating-up}.
Jak bylo popsáno v~této kapitole, na~řídící desce galvanometrů je~připevněna hliníková destička, která chladí čipy. Chladič  byl~připevněn právě na~ni.

Aktivní chladič\footnote{Chladič s~větrákem, který aktivně vytváří proud vzduchu.} se ~ale~hodí i ~pro~ostatní součástky. Vzduch, který nasaje, je~totiž distribuován celou vnitřní konstrukcí projektoru a~chladí  tak~všechny vnitřní součástky. Tomuto proudění byla věnována zvláštní pozornost při designu konstrukce pouzdra.

\subsubsection{Přístup k~portům Raspberry Pi}
Aby bylo možné je~používat, je~potřeba zajistit jednoduchý přístup  k~portům Raspberry Pi. Dále je~potřeba od~nich odlišit nabíjecí port.

\subsubsection{Modularita, jednoduchá konstrukce}
Pouzdro bylo designováno, také  aby~bylo modulární.  Aby~bylo možné při prototypování vyměnit pouze jednu součástku, která nesedí, místo tisknutí celého pouzdra od~začátku. S~modularitou bylo zároveň dosaženo jednoduché konstrukce, v~jakékoliv části stavby je~možné dočasně odstranit díly,  aby~bylo možné upravit připevnění vnitřní elektroniky.

\subsection{Konstrukce}
Pouzdro se~skládá ze~čtyř vertikálních stěn s~otvory, do~kterých zapadají horizontální desky. Horizontální desky v~sobě mají vždy z~vrchní strany vyhloubené prohlubně, do~kterých pasují elektronické součástky, které na~nich leží. Díky tomu se~elektronické součástky při pohybu projektoru volně nepohybují ve~vnitřních prostorách.

\begin{figure}[htb]
  \centering
  \includegraphics[width=1\textwidth]{img/case-sideview.jpg}
  \caption{\label{fig:case-sideview} Pohled do~projektoru  s~odstraněnou přední stěnou  v~programu Autodesk Fusion}
\end{figure}

Tyto prohlubně jsou hlavní důvod, proč  byl~k~výrobě dílů využit 3D tisk místo například laserového řezání.

Elektronika je ~v~prohlubních často držena  i~lepidlem z~tavné pistole.
To bylo přidáno  s~původním cílem upevnit  k~elektronice kabely z~ní vedoucí  i~za~izolaci. Kdyby kabely držely  jen~za~vodičové drátky  k~elektronice připájené, drátky by~se ~v~průběhu času kvůli vibracím při přenášení polámaly.

Jak je~vidět na~obrázku~\ref{fig:case-sideview}, pouzdro je~rozděleno do~pěti pater. Jednotlivá patra nezabírají celou horizontální plochu projektoru. Často spojují pouze tři ze~čtyř stěn, nebo jsou  v~nich otvory, kterými může proudit vzduch.  V~prvním patře se~nachází baterie (zvýrazněny modře), destička  s~BMS obvodem a ~v~neposlední řadě Raspberry Pi.
Na~něm je~připevněná  HAT~deska plošných spojů, která zasahuje do~druhého patra. Druhé patro je~vidět na~obrázku~\ref{fig:hw_layer0} a~nachází se ~v~něm obvod nabíjení baterie (PD trigger deska, step up~měnič a~nabíječka Li-ion článků).
Dále se ~v~něm nachází step down měnič napájející Raspberry Pi~a~step up~měnič napájející galvanometry.
\fxnote{tecka @ obr.\ref{fig:hw_layer0}} 
\begin{figure}[htb]
  \centering
  \includegraphics[width=0.8\textwidth]{img/hw_layer0.jpg}
  \caption{\label{fig:hw_layer0} Kompletně nainstalované druhé patro (Na obrázku je~modul TP5100 zapojen  s~opačnou polaritou, ve~výrobku byla chyba opravena,  ale~tato fotka je~stále nejlepší ilustrace.)}
\end{figure}

Ve třetím patře je~upeněn chladič, foukající na~nabíjecí obvod a~step up~měniče  pod~ním,  set~galvanometrů (zvýrazněn žlutě), laserový modul (zvýrazněn červeně) a~jeho řídící deska (zvýrazněna zeleně), step up~měnič  pro~laser a~relé modul.  
Galvanometry zasahují až do~čtvrtého patra,  kde~je~upevněna jejich řídící deska. Ta ~je~mimo jiné připevněna na~chladič párem šroubků viditelných na~obrázku~\ref{fig:hw_galvoboard} ze~strany \pageref{fig:hw_galvoboard}, mezi desku  a~chladič byla aplikována teplovodivá pasta.  V~prostorách čtvrtého patra je~také na~stěnu upevněna kamera.
V pátém patře se~nachází  LCD~a~rotační enkodér.

V horizontálních deskách oddělujících patra od~sebe jsou otvory  pro~vzduch,  jak~je~zmíněno  v~sekci~\ref{sec:krabick-design-priorities}, také na~přední stěně pouzdra jsou otvory  pro~vyfukování vzduchu. Na~přední stěně je~také otvor  pro~přístup  k~potenciometrům na ~HAT~DPS a~otvor  pro~přístup  k~portům Raspberry Pi,  ten~je ~i~na~pravé boční stěně.
Na~zadní stěně je~otvor  pro~nasávání vzduchu větrákem, vypínač, a~otvory  pro~USB-C port PD~trigger desky a~statusové diody nabíječky TP5100. Na~levé boční stěně jsou  pak~otvory  pro~kameru a~laserový výstup galvanometrů. Ve~dně jsou otvory  pro~výmněnu baterií a~ve~stropní stěně jsou otvory  pro~LCD a~rotační enkodér.

\begin{figure}[htb]
  \centering
  \includegraphics[width=0.8\textwidth]{img/hw_sides_backleft.jpg}
  \caption{\label{fig:hw_sides_backleft.jpg} Pohled na~projektor ze~strany hrany sousedící se~zadní a~pravou boční stěnou}
\end{figure}

\fxnote{fotka zepredu a~zprava}

