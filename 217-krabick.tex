\section{Pouzdro}

V~rámci popularizace technologie se~může hodit projektor předvádět na~různých místech. Proto bude potřeba, aby byl projektor přenosný a~při přemisťování se~nerozbil.

Pouzdro je~tedy navrženo tak, aby bylo odolné proti nárazům a~zachovalo všechny součásti v~bezpečí. Každá součástka má své místo, kde je~držena ze~všech stran. Pro výrobu pouzdra by~bylo vhodné využít technologii 3D tisku. Ta~umožňuje tvorbu komplexních geometrických tvarů přímo pro potřeby konkrétního modelu a~zároveň umožňuje snadnou iteraci a~úpravu designu pouzdra při nalezení chyb.

\subsection{Priority designu} \label{sec:krabick-design-priorities}
\subsubsection{Chlazení}
Největší část projektoru je~hliníkový chladič s~větrákem. Už od~začátku práce byl tento chladič vybrán, aby byl připevněn k~řídící desce galvanometrů. Problematika jejího zahřívání je~popsaná v~kapitole \ref{sec:galvoboard-chips-heating-up}.
Jak bylo popsáno v~této kapitole, na~řídící desce galvanometrů je~připevněna hliníková destička, která chladí čipy. Chladič byl připevněn právě na~ni.

Aktivní chladič\footnote{chladič s~větrákem, který aktivně vytváří proud vzduchu} se~ale hodí i~pro ostatní součástky. Vzduch, který nasaje je~totiž distribuován celou vnitřní konstrukcí projektoru a~chladí tak všechny vnitřní součástky. Tomuto proudění byla věnována zvláštní pozornost při designu konstrukce pouzdra.

\subsubsection{Přístup k~portům Raspberry Pi}

\fxnote {TODO: +oddělení nabíjecího portu od~nich}

\subsubsection{Modularita, jednoduchá konstrukce}
Pouzdro bylo designováno také aby bylo modulární. Aby bylo možné při prototypování vyměnit pouze jednu součástku, která nesedí místo tisknutí celého pouzdra od~začátku. S~modularitou bylo zároveň dosaženo jednoduché konstrukce, v~jakékoliv části stavby je~možné dočasně odstranit díly, aby bylo možné upravit připevnění vnitřní elektroniky.

\subsection{Konstrukce}
Pouzdro se~skládá ze~čtyř vertikálních stěn s~otvory, do~kterých zapadají horizontální desky. Horizontální desky v~sobě mají vždy z~vrchní strany vyhloubené prohlubně, do~kterých pasují elektronické součástky, které na~nich leží. Ze~spodí strany pak mají výstupky, které přidržují elementy na~deskách pod nimi. Díky tomu se~elektronické součástky při pohybu projektoru volně nepohybují ve~vnitřních prostorách.

Tyto prohlubně a~výstupky jsou hlavní důvod, proč byl k~výrobě dílů využit 3D tisk místo například laserového řezání.
% (v odstavci vys zmineno) I~horizontální desky na~sobě musí mít vertikální elementy, které přidržují součástky.

\fxnote {TODO: PHOTOSS}


