\chapter{Laserová projekce~\cite{laser-projection}}
Laserová projekce spadá mezi laser scanning technologie. Často je~využívána v zábavním průmyslu, hlavně k vytváření laser shows a~vektorových projekcí. U laser shows diváci sledují vzory, které vytváří samotný paprsek ve~vzduchu. U projekcí diváci sledují obrazce vykreslené paprskem dopadajícím na~promítací plátno.

Tyto efekty nejsou populární pouze v klubech, nýbrž i na~koncertech nebo živých představeních. Je-li laser dostatečně silný, je~možné promítat na~obrovské plochy, jako například hráze, vodní plochy, nebo dokonce hory.

\section{Využití laserové projekce v~průhledových displejích~(HUD)~\cite{laser-huds}\cite{dev-of-laser-huds-in-driving}}
Laserová projekce se~také využívá v průhledových displejích, jak bylo zmíněno v úvodu. V tomto odvětví zatím převládají jiné technologie. Technologie průhledových displejů se~dají rozdělit následovně:

\begin{itemize}
  \item Technologie vyzařujících displejů,~např. cathode ray tube~(CRT), organic light-emitting diode~(OLED) nebo vacuum fluorescent display (VFD)
  \item Technologie podsvícených displejů,~např. liquid crystal display (LCD)
  \item Technologie laserových displejů,~např. liquid crystal on~silicon (LCoS) nebo laser scanning displeje založené na~pohybu mikrozrcadel.
\end{itemize}

V prvních průhledových displejích bylo využito CRT. Ale kvůli své neskladnosti, vysoké spotřebě elektřiny a~škodlivé radiaci, byla nahrazena technologií LCD. Dnes se~v průhledových displejích letadel využívají LCD. V automobilech se~využívají LCD a~VFD.
Bohužel, VFD jsou limitovány množstvím informací, které mohou zobrazit. LCD průhledové displeje jsou limitovány svým maximálním jasem.
S novou technologií OLED sice je~možné vytvořit tenký a~průhledný displej dosahující vyšího jasu, než LCD HUD.
I tento displej bohužel oproti vnějšímu světu má stále relativně nízký jas, také má vysokou cenu a~krátkou životnost.
Oproti vyzařujícím a~podsvíceným displejům jsou laserové displeje ve~většině ohledů lepší. \fxnote{nadřazené?}

\fxnote{pics from \cite{dev-of-laser-huds-in-driving}}

\section{Princip laserové projekce}\label{sec:projection-princip}
Když se~laserový paprsek pohybuje dostatečně rychle, lidské oko ho~vnímá jako spojitou linku světla - tomuto jevu se~říká persistance of~vision nebo persistance of~impression~\cite{persistance-of-vision}.
Čím rychleji se~paprsek pohybuje, tím méně intenzivní připadá oku zmíněná linka. Bod je~možné vykreslit, jestliže paprsek zůstane na~jednom místě bez pohybu po~úrčitou dobu.

Vykreslení čáry je~základní a~nejjednoduší operací, jakou laserový projektor může vykonat. Například k vykreslení Úsečky z~bodu A~do~bodu B projektor nasměřuje laserový paprsek na~bod A, zapne laser a~pohybuje paprskem k bodu B.

K vykreslení složitějších obrázků jsou potřeba tzv. blank\ lines, kdy projektor otáčí zrcátky stejně jako kdyby vykresloval přímku, ale laser nesvítí. Blank lines spojují každé dvě vykreslené linky, které na~sebe přímo nenavazují.

Nestihne-li projektor vykreslovat obraz dostatečně rychle, výsledná projekce nebude stabilní. Lidské oko vždy uvidí pouze části obrazu v časové návaznosti. Tomuto jevu se~říká \uv{flickering}.

\section{ILDA}
