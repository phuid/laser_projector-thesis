% !TeX root = text.tex
\chapter{hardware}
Tato kapitola se~zabývá fyzickou konstrukcí a~zapojením vyrobeného laserového projektoru. Ten se~skládá z~řídící jednotky, galvanometrů s~ovládací elektronikou, laseru, chlazení a~napájení. Všechny tyto součástky jsou uloženy v~pouzdře vytisknutém na~3D tiskárně.

\section{Řídící jednotka --- Raspberry Pi}
Jako hlavní řídící jednotka byl použit jednodeskový počítač Raspberry Pi, který je vidět na obrázku~\ref{fig:RPi}. To hned z několika důvodů:
\begin{itemize}
  \item Jednoduché připojení k Wi-Fi sítím --- Raspberry Pi je možné připojit k blízké Wi-Fi síti nebo si může vytvořit vlastní, na které funguje jako wifi modem.
  \item Operační systém Raspberry Pi OS --- Běží na něm operační systém založen na linuxu, je pro něj jednoduché programovat programy a také programy můžou jednoduše interagovat s potenciálně připojenou klávesnicí, myší nebo monitorem.
  \item Vysoký výkon --- Oproti mikrokontrolerům nabízí podstatně vyšší výkon, ten se může hodit, potřebujeme-li spouštět několik programů zároveň nebo chceme-li procesovat video z kamery.
\end{itemize}
Raspberry Pi na sobě má 40 pinový GPIO header, skrze který je možné interagovat s připojenými čipy a moduly.

\begin{figure}[htb]
    \centering
  \includegraphics[width=0.8\textwidth]{img/RPi.png}
  \caption{\label{fig:RPi} Pohled zvrchu na Raspberry Pi; Převzato z~\cite{rpi-image}}
\end{figure}

\section{Set galvanometrů se~zrcátky} \label{sec:my-galvos}
\subsection{Výběr skeneru}
Pro tuto práci byl vybrán galvanometrový skener, protože je~nejdostupnější a~protože potenciálním uživatelům nejlépe představí technologii.

Oproti hranolovým skenerům jim tožiž dává více možností, jak s paprskem pohybovat.
Můžou se~rozhodnout, že jej využijí jako hranolový skener, pokud nahrají soubor procházející promítací plochu po~řádcích.

Oproti dalším typům skenerů je~názornější, ostatní typy skenerů jsou totiž příliš malé a~není na~nich vidět princip funkce nebo je~jejich fungování nadmíru abstraktní a~těžko pochopitelné.

\subsection{Zapojení galvanometrového setu}
Samotné galvanometry jsou zapojeny do~řídící desky, která s nimi byla zakoupena, ta je vidět na obrázku~\ref{fig:hw_galvoboard}.

Řídící deska požaduje symetrický zdroj napětí 15~V, tzn. $+15$~V a~$-15$~V a~samozřejmě připojení k zemi. Také přijímá dva bipolární diferenciální analogové signály s~rozsahem diferenciálního napětí $-10$~V až $+10$~V. Každý signál udává vychýlení jednoho ze~dvou galvanometrů, což obvykle znamená výslednou pozici laserového paprsku v~osách X~a~Y.

\begin{figure}[htb]
  \centering
  \includegraphics[width=1\textwidth]{img/hw_galvoboard.jpg}
  \caption{\label{fig:hw_galvoboard} Řídící deska galvanometrů s vyznačenými konektory a hřejícími čipy}
\end{figure}

\subsection{bipolární diferenciální analogový signál~\cite{ilda-signal-spec}}
Diferenciální signál je~signál přenášený dvěma vodiči, každý z~nich přenáší stejný signál, jen s~opačnou polaritou. Kontakt označený $(+)$ je~považován za~nosič základního signálu, zatímco kontakt označený $(-)$ je~považován za~nosič invertovaného signálu. Výsledné diferenciální napětí je~napětí na~základním nosiči vůči napětí na~obráceném nosiči, tzn.~$V_{dif} = V_{(+)} - V_{(-)}$

Bipolární signál znamená, že na~napětí každém z~kontaktů $(+)$ a~$(-)$ může dosahovat kladných i~záporných hodnot.

Tudíž cheme-li disáhnout diferenciálního napětí $+10~V$, musí mít základní signál napětí $+5~V$ a~obrácený signál $-5~V$. Záporné diferenciální napětí bude ve~chvíli, kdy je~napětí základního signálu záporné a~napětí obráceného signálu kladné.

\subsection{Zahřívání čipů řídící desky galvanometrů} \label{sec:galvoboard-chips-heating-up}
Dva z~čipů na~řídící desce při chodu systému výrazně zahřívájí. Na~tyto čipy naštěstí už od~výroby desky je~připevněna malá hliníková destička. Ta~má sloužit jako chladič, ale i~s ní se~čipy v~otevřeném prostoru zahřívají na~teploty blízké 60~\degree{}C.
Dva zmíněné čipy jsou čipy TDA2030A od~firmy STMicroelectronics. Ty~by~měly dle datasheetu vydržet až 150~\degree{}C, ale dá se~předpokládat, že v~uzavřeném pouzdru budou čipy dosahovat vyšších teplot, než v~otevřeném prostoru. I~kdyby nedosáhly pro sebe kritických 150~\degree{}C, rozhodně není žádoucí, aby uvnitř projektoru desky dosahovaly vysokých teplot.

I proto byl do~projektoru zabudován chladič, Více o~způsobu jeho připevnění a~distribuci chlazení mezi ostatní komponenty se~dočtete v~kapitole~\ref{sec:krabick-design-priorities}.


\section{laser}
Jako zdroj laserového paprsku byl využit RGB laserový modul, skládající se ze tří barevných diod a dichroických zrcátek.

\subsection{Dichroická zrcadla~\cite{dichronic-mirrors}}

\fxnote{TODO: slouzi k}

Dichroická zrcadla jsou zrcadla s výrazně rodílnými odrazovými nebo průchodovými vlastnostmi pro dvě různé vlnové délky odraženého~/~procházejícího světla.

Většina dichroických zrcadel jsou dielektrická zrcádla \footnote{Dielektrická zrcadla jsou zrcadla, skládající se z mnoha tenkých vrstev různě opticky propustných materiálů.}, existují ale také krystalická zrcadla\footnote{Krystalická zrcadla jsou zrcadla, jejiž odrážlivá vrstva se skládá z monokrystalického materiálu, typicky polovodiče.}.


\section{LCD displej}
\section{rotační enkodér}

\section{HAT}
Pro ovládání výše popsaného hardwaru je~zapotřebí několik specifických obvodů.
Kvůli jejich specifičnosti tyto obvody nejsou volně dostupné k~zakoupení na~předem vytvořených destičkách. Proto bylo zapotřebí je~z jednotlivých součástek vyrobit na~míru.

Obvody byly navrženy v~programu KiCad...\fxnote{bud spojit vety, nebo k~prvni neco jeste dopsat}
Následně pro ně v~tomtéž programu byla nadesignována deska plošných spojů. Na~této desce se~vyskytují obvody \fxnote{todo dac+amps, bat\_probe, -15V}.
Kromě nich byly na~desku přidány konektory k~jednotlivým barevným vstupům laseru, LCD displeji a~k rotačnímu enkodéru, které jsou přímo napojeny na~40 pinový GPIO konektor Raspberry Pi.
Deska byla designována jako tzv. HAT, to~znamená, že sama na~tomto konektoru drží a~nezabírá o~moc víc místa, než samotné Raspberry Pi.
\fxnote{TODO: obrazek desky (maybe mounted)}

\subsection{Zdroj $-15$~V}

\subsection{obvod pro generování analogového signálu}
Jak popsáno v~sekci \ref{sec:my-galvos}, řídící deska galvanometrů přijmá dva bipolární diferenciální analogové signály v~rozpětí $-5$~V až $+5$~V.

Obvod, který se~stará o~vytváření tohoto signálu je~založený na~obvodu ze~zdroje~\cite{lasershow-with-real-galvos}.
Vytváření tohoto signálu je~rozděleno do~dvou částí. Nejdříve DAC (digital-to-analog converter, D/A převodník) připojený k~RPi vytvoří signál v~rozpětí 0 až 5~V a~následně je~tento signál pomocí operačního zesilovače převeden na~požadované rozpětí, tj. $-15$~V až $+15$~V.
Jednotlivé části tohoto obvodu jsou blíže popsány v~následujících kapitolách. Celé zapojení je~vidět na~obrázku \ref{fig:dac_board}.
\fxnote{unreadable text, make schem more compact}
\begin{figure}[!htb]
  \centering
  \includegraphics[width=1\textwidth]{img/dac_board.png} 
  \caption{\label{fig:dac_board}Zapojení DAC a~zesilovačů k~RPi a~řídící desce galvanometrů}
\end{figure}

\subsubsection{dac\cite{mcp4822-dsh}}
K generování signálu v~rozpětí 0--5~V byl využit dvoukanálový D/A převodník\footnote{obvod, který na~základě instrukcí přijatých digitálně generuje analogové napětí} MCP4822.
Tento čip podporuje komunikaci přes rozhraní SPI, pracuje s~napájecím napětím 5~V a~s 12bitovým rozlišením (je schopen vygenerovat 4~096 různých napětí) na~dvou kanálech.

RPi komunikuje s~čipem pomocí rozhraním SPI.
\fxnote{TODO more spec}
Tato knihovna poskytuje následující funkce, se~kterými pracuji v~mém kódu.
\begin{itemize}
\item
\lstinline[language=C]!bool mcp4822_initialize();!
\item
\lstinline[language=C]!bool mcp4822_set_voltage(mcp4822_channel_t channel, uint16_t value_mV);!
\item
\lstinline[language=C]!void mcp4822_deinitialize();!
\end{itemize}
\subsubsection{amps\cite{tl082-dsh}}
K modifikaci signálu z~DAC na~bipolární diferenciální analogový signál slouží pro každý kanál jeden čip TL082, který obsahují dva operační zesilovače. Ty~jsou zapojeny dle schématu na~obrázku \ref{fig:ilda_amps-scheme}.

Signál první operační zesilovač zesílí a~posune dle nastavení potenciometrů Ygain(zesílení) a~Yoffset(posun) a~zároveň invertuje. Tento invertovaný signál následně druhý operační zesilovač opět invertuje, získav základní signál pro řídící desku galvanometrů.

\begin{figure}[!htb]
  \centering
  \includegraphics[width=1\textwidth]{img/ilda_amps.png} 
  \caption{\label{fig:ilda_amps-scheme} Zapojení čipu TL082 pro jeden kanál řídící desky galvanometrů}
\end{figure}

\fxnote{TODO more spec}
Tyto čipy mi~napěťové rozpětí zvýší z~0--5~V na~$-15$~V až $+15$~V.

zesilovac - cteni baterek \url{https://is.muni.cz/el/sci/jaro2017/F5090/um/E17_P8.pdf}


\section{Napájení}
Celkové schéma napájení je~vidět na~obrázku~\ref{fig:power-schem-full}.

\begin{figure}[!h]
  \centering
  \includegraphics[width=\textwidth]{img/power-schem-full.jpg}
  \caption{\label{fig:power-schem-full} Celkové schéma napájení komponentů projektoru}
\end{figure}

\subsection{Akumulátory}
K napájení projektoru byly využity 4 Lithium-iontové akumulátory Samsung INR~18650 s~kapacitou 3~450 ~mAh~a jmenovitým napětím 3,7~V. Ty~byly zapojeny nejdříve po~dvojicích paralelně a~následně byly tyto dvojice zapojeny sériově. Konečný článek tedy dosahuje jmenovitého napětí 7,4~V.

\subsection{BMS modul}
Na baterie byl~napojen ochranný BMS~(Battery management system) modul, který ji~chrání před následujícími stavy:
\begin{itemize}
  \item odběr vysokého proudu (zkrat),
  \item přebití,
  \item vybití,
  \item nevybalancované články.
\end{itemize}
Modul články balancuje a~v případě, že nastane jiný z~nežádoucích stavů, ji~odpojí. Modul je ~na~obrázku~\ref{fig:BMS}


\begin{figure}[htb]
  \centering
  \begin{minipage}{0.45\textwidth}
    \centering
  \includegraphics[width=0.8\textwidth]{img/BMS.jpg}
  \caption{\label{fig:BMS} BMS~Modul se~třemi kontakty pro~sérii baterií (0V, 4.2V a~8.4V) a~výstupními kontakty ($(+)$ a~$(-)$)~\cite{laskakit-BMS}}
  \end{minipage}\hfill
  \begin{minipage}{0.45\textwidth}
    \centering
  \includegraphics[width=0.8\textwidth]{img/relay.jpg}
  \caption{\label{fig:relay} Relé modul~\cite{laskakit-relay}}
  \end{minipage}
\end{figure}

\subsection{Relé modul}
V projektoru byl~využit relé modul pro~připojování komponentů s~vysokým odběrem pouze ve~chvílích, kdy~jsou využívány. Jedná se ~o~set galvanometrů, laserový modul a~větrák, ty~jsou připojeny jen~ve~chvílích, kdy~projektor promítá.

Relé modul je~ovládán jedním kontaktem spojeným s~Raspberry Pi ~a~je~umístěn mezi bateriemi a~měniči napětí, proto stačí pouze jeden na~více napěťových větví. Je~vidět na~obrazku~\ref{fig:relay}

\subsection{Nabíjecí obvod}
Zapojený nabíjecí obvod je~vidět na~obrázku~\ref{fig:hw_charging_circuit}.

\begin{figure}[htb]
  \centering
  \includegraphics[width=0.8\textwidth]{img/hw_charging_circuit.jpg}
  \caption{\label{fig:hw_charging_circuit} Zapojený nabíjecí obvod}
\end{figure}

\subsubsection{Power Delivery (PD) trigger deska}
K nabíjení baterie je~využíván USB-C port podporující moderní protokoly rychlého nabíjení (hlavně Power Delivery a~Quick Charge).
Ten se~nachází na~desce s~integrovaným obvodem, který přes port komunikuje s~adaptérem, pokud adaptér podporuje rychlé nabíjení, čip od~něj vyžádá napětí 12~V, které deska převádí na~výstupní kontakty viditelné na~obrázku~\ref{fig:PDtrig}. Protože deska \uv{vyvolá} dané napětí, označuje se ~PD~trigger board (deska).

\begin{figure}[htb]
  \centering
  \begin{minipage}{0.3\textwidth}
    \centering
  \includegraphics[width=1\textwidth]{img/PDtrig.jpg}
  \caption{\label{fig:PDtrig} Power Delivery trigger board~\cite{laskakit-PD}}
  \end{minipage}\hfill
  \begin{minipage}{0.3\textwidth}
    \centering
  \includegraphics[width=1\textwidth]{img/XL6009.jpg}
  \caption{\label{fig:XL6009} step-up měnič s~čipem XL6009~\cite{laskakit-XL6009}}
  \end{minipage}
  \begin{minipage}{0.3\textwidth}
    \centering
    \includegraphics[width=0.8\textwidth]{img/TP5100.jpg}
    \caption{\label{fig:TP5100} Modul nabíječky dvou sériově zapojených Li-ion baterií~\cite{laskakit-TP5100}}
  \end{minipage}
\end{figure}

\subsubsection{Step up~měnič s~čipem XL6009}
Pokud ovšem připojený adaptér nepodporuje žádný rychlonabíjecí protokol, na~výstupech desky bude napětí pouze 5~V, na~kterém standartně běží USB~připojení. Proto je ~k~PD~trigger board připojen step-up měnič nastavený na~9~V.
Pokud z~PD~desky bude vycházet napětí 5~V, step-up jej~zvýší na~9~V. Pokud z~PD~desky bude vycházet 12~V, napětí step-up projde beze změny.

\subsubsection{TP5100}
K ovládání průběhu nabíjení byl~využit modul pro~nabíjení Li-ionových baterií s~čipem TP5100. Ten~zajišťuje konstantní proud a~napětí, které posílá na~kontakty BMS~obvodu. Další z~jeho funkcí je~automatické ukončení nabíjení ve~chvíli, kdy~baterie dosáhnou napětí 8,4~V. Je ~to~jedinečný modul, který umožňuje nabíjení dvoučlánkových lithium-iontových akumulátorů.

\subsection{Vypínač}
K bateriím je~neustále připojený jen~BMS modul a~nabíjecí obvod, všechny ostatní obvody jsou přemostěny vypínačem. Je~tedy možné baterie nabíjet, i~když jsou všechny ostatní obvody odpojené. Vypínač je~vidět na~obrázku~\ref{fig:hw_charging_circuit}.

\subsection{Napěťové větve}
Různé komponenty projektoru pracují s~různými napětími. Je~tedy potřeba napětí baterií převést na~několik napěťových větví. Jedná se ~o~větve:

\begin{itemize}
  \item 5V --- Napětí Raspberry Pi, LCD~a relé modulu; Zajištěno step-down měničem s~čipem XL4005 (viz obrázek~\ref{fig:XL4005}.)
  \item 12V --- Napětí Laserového modulu; Zajištěno step-up měničem s~čipem MT3608 (Viz obrázek~\ref{fig:MT3608}.)
  \item Symetrické napětí $\pm{}15$ ~V~--- Napětí řídící desky galvanmetrů; Kladná větev je~zajištěna step-up měničem s~čipem XL6009 (viz obrázek~\ref{fig:XL6009}.), záporná větev je~zajištěna obvodem zdroje -15 ~V~na ~HAT~DPS (viz kapitola~\ref{sec:negative-ps}).
\end{itemize}


\begin{figure}[htb]
  \centering
  \begin{minipage}{0.45\textwidth}
    \centering
  \includegraphics[width=0.8\textwidth]{img/XL4005.jpg}
  \caption{\label{fig:XL4005} step-down měnič s~čipem XL4005~\cite{laskakit-XL4005}}
  \end{minipage}\hfill
  \begin{minipage}{0.45\textwidth}
    \centering
  \includegraphics[width=0.8\textwidth]{img/MT3608.jpg}
  \caption{\label{fig:MT3608} step-up měnič s~čipem MT3608~\cite{laskakit-MT3608}}
  \end{minipage}
\end{figure}


\section{Pouzdro}

V~rámci popularizace technologie se~může hodit projektor předvádět na~různých místech. Proto bude potřeba, aby byl projektor přenosný a~při přemisťování se~nerozbil.

Pouzdro je~tedy navrženo tak, aby bylo odolné proti nárazům a~zachovalo všechny součásti v~bezpečí. Každá součástka má své místo, kde je~držena ze~všech stran. Pro výrobu pouzdra by~bylo vhodné využít technologii 3D tisku. Ta~umožňuje tvorbu komplexních geometrických tvarů přímo pro potřeby konkrétního modelu a~zároveň umožňuje snadnou iteraci a~úpravu designu pouzdra při nalezení chyb.

\subsection{Priority designu} \label{sec:krabick-design-priorities}
\subsubsection{Chlazení}
Největší část projektoru je~hliníkový chladič s~větrákem. Už od~začátku práce byl tento chladič vybrán, aby byl připevněn k~řídící desce galvanometrů. Problematika jejího zahřívání je~popsaná v~kapitole \ref{sec:galvoboard-chips-heating-up}.
Jak bylo popsáno v~této kapitole, na~řídící desce galvanometrů je~připevněna hliníková destička, která chladí čipy. Chladič byl připevněn právě na~ni.

Aktivní chladič\footnote{chladič s~větrákem, který aktivně vytváří proud vzduchu} se~ale hodí i~pro ostatní součástky. Vzduch, který nasaje je~totiž distribuován celou vnitřní konstrukcí projektoru a~chladí tak všechny vnitřní součástky. Tomuto proudění byla věnována zvláštní pozornost při designu konstrukce pouzdra.

\subsubsection{Přístup k~portům Raspberry Pi}

\fxnote {TODO: +oddělení nabíjecího portu od~nich}

\subsubsection{Modularita, jednoduchá konstrukce}
Pouzdro bylo designováno také aby bylo modulární. Aby bylo možné při prototypování vyměnit pouze jednu součástku, která nesedí místo tisknutí celého pouzdra od~začátku. S~modularitou bylo zároveň dosaženo jednoduché konstrukce, v~jakékoliv části stavby je~možné dočasně odstranit díly, aby bylo možné upravit připevnění vnitřní elektroniky.

\subsection{Konstrukce}
Pouzdro se~skládá ze~čtyř vertikálních stěn s~otvory, do~kterých zapadají horizontální desky. Horizontální desky v~sobě mají vždy z~vrchní strany vyhloubené prohlubně, do~kterých pasují elektronické součástky, které na~nich leží. Ze~spodí strany pak mají výstupky, které přidržují elementy na~deskách pod nimi. Díky tomu se~elektronické součástky při pohybu projektoru volně nepohybují ve~vnitřních prostorách.

Tyto prohlubně a~výstupky jsou hlavní důvod, proč byl k~výrobě dílů využit 3D tisk místo například laserového řezání.
% (v odstavci vys zmineno) I~horizontální desky na~sobě musí mít vertikální elementy, které přidržují součástky.

\fxnote {TODO: PHOTOSS}



