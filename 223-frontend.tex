\section{UI}
\fxnote{why is wiringpi used - rewrite to pigpio is comng}
Program UI~je~také psaný v~jazyce c++ a~využívá knihovnu WiringPi, která umožňuje jednoduchou komunikaci s~GPIO piny Raspberry Pi. Tento program ovládá OLED displej, který je~připojený na~Raspberry Pi~pomocí rozhraní I2C, a~přijímá vstup od~uživatele čtením rotačního enkodéru s~tlačítkem.

Program se~při začátku exekuce pomocí knihovny ZeroMQ přihlásí ke~vstupnímu socketu a~k odběru zpráv z~výstupního TCP socketu, kam publikuje zprávy o~stavu vykreslování program lasershow. Dále si~pomocí knihovny wiringPi zaregistruje zpracovávání přerušení z~enkodéru a~tlačítka na~něm a~čeká buď na~interakci s~uživatelem, který by~skrz něj poslal zprávy programu lasershow, nebo na~zprávy od~lasershow, které by~zobrazil uživateli.

\fxnote{TODO: diagram programu}

\section{web\_ui}

Narozdíl od~předchozích dvou zmiňovaných programů je~program web\_ui psaný v~jazyce javascript, ten nepatří mezi nejrychlejší, ale díky runtime Node.js a~knihovnám http a~formidable v~něm bylo časově nenáročné vytořit http web server.

Tento server běží na~portu 3000 a~je~dostupný z~lokální sítě (tzn. přímo z\~Raspberry Pi~na~adrese http://localhost:3000 nebo z~jakéhokoliv zařízení na~stejné lokální síti na~ip~adrese RPi).
Program je~využíván pro jednoduchou interakci s~uživatelem, který může pomocí webového prohlížeče ovládat laserový projektor pár kliknutími i~zadávat vlastní příkazy klávesnicí.

\fxnote{na webu jsou konzole pro ssh, wifiman a~lasershow, taky fast project forms}

\fxnote{TODO: příklad http serveru}
\inputminted{js}{code_examples/http_static_files.js}

Stejně jako program UI~za~pomoci knihovny ZeroMQ tento program odebírá z~výstupního socketu zprávy o~průběhu vykreslování od~programu lasershow a~odesílá mu~pokyny uživatele na~vstupní socket.

\fxnote{TODO: příklad přihlášení k~socketům v~js}

\fxnote{TODO: xterm + ssh}

\section{discord bot}

Posledním programem, který je~využíván k~interakci s~uživatelem je~discord\_bot, který je~také psaný v~jazyce javascript v~runtime Node.js, stejně jako předchozí programy se~přihlásí k~socketům knihovnou zmq, ale na~rozdíl od~nich tento program může interagovat s~uživatelem přes internet ať už je~kdekoliv na~světě.
Pomocí knihovny discord.js se~přihlásí k~předem vytvořenému bot účtu, který může na~předem vytvořeném discord serveru čekat na~zprávy od~uživatele, ty~posílat do~vstupního socketu a~posílat uživateli zpětnou vazbu, kterou příjme z~výstupního socketu.

