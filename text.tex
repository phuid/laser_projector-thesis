% !TeX root = text.tex
\documentclass{template/socthesis}

\usepackage{subcaption}
\usepackage{amsmath}
\usepackage{enumitem}

% my own code
\usepackage[skip=10pt plus1pt, indent=0pt]{parskip}

\addbibresource{text.bib}

\titlecz{Laserový projektor}
\titleen{Laser projector}
\author{Šimon Hrouda}
\field{10}
\school{Gymnázium Brno-Řečkovice}
\exmentor{Tomáš Rohlínek}
\exmentorstatement{Tomáše Rohlínka}
\inmentor{Mgr. Kateřina Vídenková}
\inmentorstatement{Mgr. Kateřiny Vídenkové}

% Změňte, pokud se liší
%\region{Jihomoravský}
\placefooter{Brno 2024}

\begin{document}

\maketitle

\makecopyrightstatement{V~Brně}

\makethanks{Děkuji svému externímu konzultantovi Tomáši Rohlínkovi a své interní konzultantce Mgr. Kateřině Vídenkové za obětavou pomoc, podnětné připomínky a nekonečnou trpělivost, kterou mi během práce poskytovali.}

\pagestyle{empty}

\section*{Anotace}


\subsection*{Klíčová slova}


\vspace{20mm}

\section*{Annotation}


\subsection*{Keywords}


\newpage
\pagestyle{plain}

\tableofcontents % vysází obsah

%%% Začátek práce
\setcounter{figure}{0}
\setcounter{table}{0}
\newpage

\chapter*{Úvod}
\addcontentsline{toc}{chapter}{Úvod} % přidá položku úvod do obsahu
V této práci se zaměřuji na návrh a výrobu laserového projektoru, který za bude za pomoci páru zrcátek připevněných na galvanometrech rsychle měnit směr laserového paprsku a tím vykreslovat obraz na promítací plochu.


\newpage

\chapter{promítání lol}


\section{Galvanometr}
https://en.wikipedia.org/wiki/Galvanometer \\
https://en.wikipedia.org/wiki/Laser\_scanning


https://en.wikipedia.org/wiki/Mirror\_galvanometer
The following is adapted from a contemporary account of Thomson's instrument:[2]\\
The mirror galvanometer consists of a long fine coil of silk-covered copper wire. In the heart of that coil, within a little air-chamber, a small round mirror is hung by a single fibre of floss silk, with four tiny magnets cemented to its back. A beam of light is thrown from a lamp upon the mirror, and reflected by it upon a white screen or scale a few feet distant, where it forms a bright spot of light. When there is no current on the instrument, the spot of light remains stationary at the zero position on the screen; but the instant a current traverses the long wire of the coil, the suspended magnets twist themselves horizontally out of their former position, the mirror is inclined with them, and the beam of light is deflected along the screen to one side or the other, according to the nature of the current. If a positive electric current gives a deflection to the right of zero, a negative current will give a deflection to the left of zero, and vice versa.
The air in the little chamber surrounding the mirror is compressed at will, so as to act like a cushion, and deaden the movements of the mirror. The needle is thus prevented from idly swinging about at each deflection, and the separate signals are rendered abrupt. At a receiving station the current coming in from the cable has simply to be passed through the coil before it is sent into the ground, and the wandering light spot on the screen faithfully represents all its variations to the clerk, who, looking on, interprets these, and cries out the message word by word. The small weight of the mirror and magnets which form the moving part of this instrument, and the range to which the minute motions of the mirror can be magnified on the screen by the reflected beam of light, which acts as a long impalpable hand or pointer, render the mirror galvanometer marvellously sensitive to the current, especially when compared with other forms of receiving instruments. Messages could be sent from the United Kingdom to the United States through one Atlantic cable and back again through another, and there received on the mirror galvanometer, the electric current used being that from a toy battery made out of a lady's silver thimble, a grain of zinc, and a drop of acidulated water.
The practical advantage of this extreme delicacy is that the signal waves of the current may follow each other so closely as almost entirely to coalesce, leaving only a very slight rise and fall of their crests, like ripples on the surface of a flowing stream, and yet the light spot will respond to each. The main flow of the current will of course shift the zero of the spot, but over and above this change of place the spot will follow the momentary fluctuations of the current which form the individual signals of the message. What with this shifting of the zero and the very slight rise and fall in the current produced by rapid signalling, the ordinary land line instruments are quite unserviceable for work upon long cables.

ovládá se variabilním proudem

\section{hlavice}
\section{řídící deska galv}
\section{moje deska na napětí}
\subsection{dac}
\subsection{amps}

\section{laser}
\section{if rgb: 3 dacs}


\chapter{ovládání}
\section{lasershow exec}
\section{lcd + encoder}
\section{web ui}
\section{discord bot}
\section{HOTSPOT}

\chapter{Diskuze}
\section{další zpracování tématu}
ze hej ze \href{https://dspace.vutbr.cz/bitstream/handle/11012/38621/final-thesis.pdf?sequence=-1}{typek z vut} udelal kinda kurva podobnej HW jak ja, ale ja to mam trochu jinak, cuz jsem o tom nevedel, ale ofc moje je lepsi :))
also to delala hromada dalsich lidi na internetu ten hw, also od gh.com/tteskac mam executable, kterou jsem ale totalne ze rozsiril a taky jsem pridal vsechno moje genialni ui muhahahah

\B{ze este dalsi zpracovani:}
\begin{enumerate}
  \item used/modified code
  \begin{itemize}
    \item \url{https://github.com/marcan/openlase/blob/master/tools/svg2ild.py}
    \item \url{https://github.com/tteskac/rpi-lasershow}
    \item \url{https://github.com/sabhiram/raspberry-wifi-conf/blob/master/app/wifi\_manager.js}
    \item \url{http://www.electronicayciencia.com/wPi\_soft\_lcd/}
    \item \href{https://dspace.vutbr.cz/bitstream/handle/11012/38621/final-thesis.pdf?sequence=-1}{typek z vut}
  \end{itemize}
  \item dalsi zpracovani stejny projekty
  \begin{itemize}
    \item \url{https://www.instructables.com/Arduino-Laser-Show-With-Real-Galvos/}
    \item \url{https://github.com/tteskac/rpi-lasershow}
    \item \url{https://www.instructables.com/DIY-STEPDIR-LASER-GALVO-CONTROLLER/}
  \end{itemize}
  \item other useful thingies
  \begin{itemize}
    \item \url{https://hackaday.io/project/172284-galvo-laser-cutterengraver}

    \item \url{https://hackaday.io/project/172284/instructions}

    \item \url{https://learn.adafruit.com/mcp4725-12-bit-dac-with-raspberry-pi/hooking-it-up}
    \item \url{https://www.ilda.com/resources/StandardsDocs/ILDA\_IDTF14\_rev011.pdf}
    \item \url{https://marcan.st/projects/openlase/}
    \item \url{https://www.youtube.com/watch?v=u9TpJ-\_hBR8}
  \end{itemize}
  \item read
  \begin{itemize}
    \item \url{https://www.laserworld.com/en/glossary-definitions/90-t/2797-ttl-modulation-en.html}
  \end{itemize}

\end{enumerate}

\newpage
\chapter*{Závěr}
\addcontentsline{toc}{chapter}{Závěr} %//FIXME proc vsichni maji zaver v obsahu jako section, kdyz pak vypada, ze je pod posledni kapitolou??
%//TODO závěr
proc vsichni maji zaver v obsahu jako section, kdyz pak vypada, ze je pod posledni kapitolou

\newpage
\printbibliography[title=Literatura]
\addcontentsline{toc}{section}{Literatura}

\listoffigures
\addcontentsline{toc}{section}{Seznam obrázků}

\listoftables
\addcontentsline{toc}{section}{Seznam tabulek}
%
% \listoflistedequation
% \addcontentsline{toc}{section}{Seznam rovnic}

\end{document}
