% !TeX root = text.tex
\documentclass{template/socthesis}

\usepackage{subcaption}
\usepackage{amsmath}
\usepackage{enumitem}

% my own code
\usepackage[skip=10pt plus1pt, indent=0pt]{parskip}

\addbibresource{text.bib}

\titlecz{Laserový projektor}
\titleen{Laser projector}
\author{Šimon Hrouda}
\field{10}
\school{Gymnázium Brno-Řečkovice}
\exmentor{Tomáš Rohlínek}
\exmentorstatement{Tomáše Rohlínka}
\inmentor{Mgr. Kateřina Vídenková}
\inmentorstatement{Mgr. Kateřiny Vídenkové}

% Změňte, pokud se liší
%\region{Jihomoravský}
\placefooter{Brno 2024}

\begin{document}

\maketitle

\makecopyrightstatement{V~Brně}

\makethanks{Děkuji svému externímu konzultantovi Tomáši Rohlínkovi a své interní konzultantce Mgr. Kateřině Vídenkové za obětavou pomoc, podnětné připomínky a nekonečnou trpělivost, kterou mi během práce poskytovali.}

\pagestyle{empty}

\section*{Anotace}


\subsection*{Klíčová slova}


\vspace{20mm}

\section*{Annotation}


\subsection*{Keywords}


\newpage
\pagestyle{plain}

\tableofcontents % vysází obsah

%%% Začátek práce
\setcounter{figure}{0}
\setcounter{table}{0}
\newpage

\chapter*{Úvod}
\addcontentsline{toc}{chapter}{Úvod} % přidá položku úvod do obsahu
V této práci se zaměřuji na návrh a výrobu laserového projektoru, který za bude za pomoci páru zrcátek připevněných na galvanometrech rsychle měnit směr laserového paprsku a tím vykreslovat obraz na promítací plochu.


\newpage
%//TODO defajnos pojmos  vvv
\chapter{hardware}
\section{promítání lol}

\subsection{Galvanometr a zrcátko}
\url{https://en.wikipedia.org/wiki/Galvanometer}

\url{https://en.wikipedia.org/wiki/Laser_scanning}

\url{https://en.wikipedia.org/wiki/Mirror_galvanometer} \\

A mirror galvanometer is an ammeter that indicates it has sensed an electric current by deflecting a light beam with a mirror.

Zrcadlový galvanometr je měřič proudu, který reaguje na měřený proud vychýlením světelného paprsku zrcátkem připevněném na jeho konci.\cite{zrcadlovy-galvanometr-wiki}

The mirror galvanometer consists of a long fine coil of silk-covered copper wire. In the heart of that coil, within a little air-chamber, a small round mirror is hung by a single fibre of floss silk, with four tiny magnets cemented to its back
The small weight of the mirror and magnets which form the moving part of this instrument, and the range to which the minute motions of the mirror can be magnified on the screen by the reflected beam of light, which acts as a long impalpable hand or pointer, render the mirror galvanometer marvellously sensitive to the current, especially when compared with other forms of receiving instruments.

ovládá se variabilním proudem

\subsection{hlavice}
\href{https://elenlaser.com/blog/galvo-head-laser-focus-tool.html}{The mirrors, mounted perpendicularly on the engines, move the laser beam along the X and Y axes according to the input received from the motor.
The big advantage of these devices is that they can reach a very high acceleration and speed of movement.}
\subsection{řídící deska galv}
\subsection{moje deska na napětí}
\subsubsection{dac}
\subsubsection{amps}

\subsection{laser}
\subsection{if rgb: 3 dacs}


%//TODO cos udelal svyho vlastne a jak to facha
\section{ovládání}
\subsection{lasershow exec}
\subsection{lcd + encoder}
\subsection{web ui}
\subsection{discord bot}
\subsection{HOTSPOT}

\section{napájení}
%//TODO ay tak co, zvladls to dat na baterky?

\chapter{software}

%//TODO udelals to vubec dobre? porovnej se s ostatnima
\chapter{Diskuze}
\section{další zpracování tématu}
udelal jsem to dobre? vybral jsem si dobry technky?
like byl by lepsi ten harddrive z yt?
nebo fakt to melo byt napajeny z baterek a ne ze zasuvky?

ze hej ze \href{https://dspace.vutbr.cz/bitstream/handle/11012/38621/final-thesis.pdf?sequence=-1}{typek z vut} udelal kinda kurva podobnej HW jak ja, ale ja to mam trochu jinak, cuz jsem o tom nevedel, ale ofc moje je lepsi :))
also to delala hromada dalsich lidi na internetu ten hw, also od gh.com/tteskac mam executable, kterou jsem ale totalne ze rozsiril a taky jsem pridal vsechno moje genialni ui muhahahah

\B{ze este dalsi zpracovani: (19.10.2023 vsechny dostupne)}
\begin{enumerate}
  \item used/modified code
  \begin{itemize}
    \item \url{https://github.com/marcan/openlase/blob/master/tools/svg2ild.py}
    \item \url{https://github.com/tteskac/rpi-lasershow}
    \item \url{https://github.com/sabhiram/raspberry-wifi-conf/blob/master/app/wifi_manager.js}
    \item \url{http://www.electronicayciencia.com/wPi_soft_lcd/}
    \item \href{https://dspace.vutbr.cz/bitstream/handle/11012/38621/final-thesis.pdf?sequence=-1}{typek z vut}
  \end{itemize}
  \item dalsi zpracovani stejny projekty
  \begin{itemize}
    \item \url{https://www.instructables.com/Arduino-Laser-Show-With-Real-Galvos/}
    \item \url{https://github.com/tteskac/rpi-lasershow}
    \item \url{https://www.instructables.com/DIY-STEPDIR-LASER-GALVO-CONTROLLER/}
    \item \href{https://youtu.be/u9TpJ-_hBR8?si=mHy-UrptZZJ0Xu5-}{borec na yt hard-drive text gut}
  \end{itemize}
  \item other useful thingies
  \begin{itemize}
    \item \url{https://hackaday.io/project/172284-galvo-laser-cutterengraver}

    \item \url{https://hackaday.io/project/172284/instructions}

    \item \url{https://learn.adafruit.com/mcp4725-12-bit-dac-with-raspberry-pi/hooking-it-up}
    \item \url{https://www.ilda.com/resources/StandardsDocs/ILDA_IDTF14_rev011.pdf}
    \item cool demos \url{https://marcan.st/projects/openlase/}
    \item \url{https://www.youtube.com/watch?v=u9TpJ-_hBR8}
  \end{itemize}
  \item read
  \begin{itemize}
    \item \url{https://www.laserworld.com/en/glossary-definitions/90-t/2797-ttl-modulation-en.html}
  \end{itemize}

\end{enumerate}

\newpage
\chapter*{Závěr}
\addcontentsline{toc}{chapter}{Závěr} %//FIXME proc vsichni maji zaver v obsahu jako section, kdyz pak vypada, ze je pod posledni kapitolou??
%//TODO závěr
proc vsichni maji zaver v obsahu jako section, kdyz pak vypada, ze je pod posledni kapitolou

\newpage
\printbibliography[title=Literatura]
\addcontentsline{toc}{section}{Literatura}

\listoffigures
\addcontentsline{toc}{section}{Seznam obrázků}

\listoftables
\addcontentsline{toc}{section}{Seznam tabulek}
%
% \listoflistedequation
% \addcontentsline{toc}{section}{Seznam rovnic}

\end{document}
