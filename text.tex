% !TeX root = text.tex
\documentclass{template/socthesis}

\usepackage{subcaption}
\usepackage{amsmath}
\usepackage{enumitem}

% my own code
\usepackage{xkvltxp}

\usepackage[skip=10pt plus1pt, indent=0pt]{parskip}
\usepackage[czech]{babel}
\usepackage{float}

\usepackage[nomargin, inline, marginclue, author=,status=draft]{fixme}
\makeatletter
\renewcommand*\FXLayoutInline[3]{%
  {\@fxuseface{inline}\ignorespaces[#2]}}
\makeatother

% \widowpenalty=5000
% \clubpenalty=5000
% \brokenpenalty=5000

%%%%%% podbarvení bloků kódu / vygenerovaných ai
\usepackage{xcolor}
\usepackage{listings}
\usepackage{tcolorbox}

\tcbuselibrary{breakable}

\definecolor{aiblue}{rgb}{127,127,255}
\definecolor{aibackground}{rgb}{100,100,100}

\lstdefinestyle{aistyle}{
    backgroundcolor=\color{aibackground},   
    commentstyle=\color{aiblue},
    keywordstyle=\color{aiblue},
    numberstyle=\tiny\color{aiblue},
    stringstyle=\color{aiblue},
    basicstyle=\ttfamily\footnotesize,
    breakatwhitespace=false,         
    breaklines=true,                 
    captionpos=b,                    
    keepspaces=true,                 
    numbers=left,                    
    numbersep=5pt,                  
    showspaces=false,                
    showstringspaces=false,
    showtabs=false,                  
    tabsize=2
}

\definecolor{codegreen}{rgb}{0,0.6,0}
\definecolor{codegray}{rgb}{0.5,0.5,0.5}
\definecolor{codepurple}{rgb}{0.58,0,0.82}
\definecolor{backcolour}{rgb}{0.95,0.95,0.92}

\lstdefinestyle{code}{
    backgroundcolor=\color{backcolour},   
    commentstyle=\color{codegreen},
    keywordstyle=\color{magenta},
    numberstyle=\tiny\color{codegray},
    stringstyle=\color{codepurple},
    basicstyle=\ttfamily\footnotesize,
    breakatwhitespace=false,         
    breaklines=true,                 
    captionpos=b,                    
    keepspaces=true,                 
    numbers=left,                    
    numbersep=5pt,                  
    showspaces=false,                
    showstringspaces=false,
    showtabs=false,                  
    tabsize=2
}
% languages
\definecolor{lightgray}{rgb}{0.95, 0.95, 0.95}
\definecolor{darkgray}{rgb}{0.4, 0.4, 0.4}
%\definecolor{purple}{rgb}{0.65, 0.12, 0.82}
\definecolor{editorGray}{rgb}{0.95, 0.95, 0.95}
\definecolor{editorOcher}{rgb}{1, 0.5, 0} % #FF7F00 -> rgb(239, 169, 0)
\definecolor{editorGreen}{rgb}{0, 0.5, 0} % #007C00 -> rgb(0, 124, 0)
\definecolor{orange}{rgb}{1,0.45,0.13}		
\definecolor{olive}{rgb}{0.17,0.59,0.20}
\definecolor{brown}{rgb}{0.69,0.31,0.31}
\definecolor{purple}{rgb}{0.38,0.18,0.81}
\definecolor{lightblue}{rgb}{0.1,0.57,0.7}
\definecolor{lightred}{rgb}{1,0.4,0.5}
\usepackage{upquote}
\usepackage{listings}
% CSS
\lstdefinelanguage{CSS}{
  keywords={color,background-image:,margin,padding,font,weight,display,position,top,left,right,bottom,list,style,border,size,white,space,min,width, transition:, transform:, transition-property, transition-duration, transition-timing-function},	
  sensitive=true,
  morecomment=[l]{//},
  morecomment=[s]{/*}{*/},
  morestring=[b]',
  morestring=[b]",
  alsoletter={:},
  alsodigit={-}
}

% JavaScript
\lstdefinelanguage{JavaScript}{
  morekeywords={typeof, new, true, false, catch, function, return, null, catch, switch, var, if, in, while, do, else, case, break},
  morecomment=[s]{/*}{*/},
  morecomment=[l]//,
  morestring=[b]",
  morestring=[b]'
}

\lstdefinelanguage{HTML5}{
  language=html,
  sensitive=true,	
  alsoletter={<>=-},	
  morecomment=[s]{<!-}{-->},
  tag=[s],
  otherkeywords={
  % General
  >,
  % Standard tags
	<!DOCTYPE,
  </html, <html, <head, <title, </title, <style, </style, <link, </head, <meta, />,
	% body
	</body, <body,
	% Divs
	</div, <div, </div>, 
	% Paragraphs
	</p, <p, </p>,
	% scripts
	</script, <script,
  % More tags...
  <canvas, /canvas>, <svg, <rect, <animateTransform, </rect>, </svg>, <video, <source, <iframe, </iframe>, </video>, <image, </image>, <header, </header, <article, </article
  },
  ndkeywords={
  % General
  =,
  % HTML attributes
  charset=, src=, id=, width=, height=, style=, type=, rel=, href=,
  % SVG attributes
  fill=, attributeName=, begin=, dur=, from=, to=, poster=, controls=, x=, y=, repeatCount=, xlink:href=,
  % properties
  margin:, padding:, background-image:, border:, top:, left:, position:, width:, height:, margin-top:, margin-bottom:, font-size:, line-height:,
	% CSS3 properties
  transform:, -moz-transform:, -webkit-transform:,
  animation:, -webkit-animation:,
  transition:,  transition-duration:, transition-property:, transition-timing-function:,
  }
}

\lstdefinestyle{htmlcssjs} {%
  % General design
%  backgroundcolor=\color{editorGray},
  basicstyle={\footnotesize\ttfamily},   
  frame=b,
  % line-numbers
  xleftmargin={0.75cm},
  numbers=left,
  stepnumber=1,
  firstnumber=1,
  numberfirstline=true,	
  % Code design
  identifierstyle=\color{black},
  keywordstyle=\color{blue}\bfseries,
  ndkeywordstyle=\color{editorGreen}\bfseries,
  stringstyle=\color{editorOcher}\ttfamily,
  commentstyle=\color{brown}\ttfamily,
  % Code
  language=HTML5,
  alsolanguage=JavaScript,
  alsodigit={.:;},	
  tabsize=2,
  showtabs=false,
  showspaces=false,
  showstringspaces=false,
  extendedchars=true,
  breaklines=true,
  % German umlauts
  literate=%
  {Ö}{{\"O}}1
  {Ä}{{\"A}}1
  {Ü}{{\"U}}1
  {ß}{{\ss}}1
  {ü}{{\"u}}1
  {ä}{{\"a}}1
  {ö}{{\"o}}1
}
%
\lstdefinestyle{py} {%
language=python,
literate=%
*{0}{{{\color{lightred}0}}}1
{1}{{{\color{lightred}1}}}1
{2}{{{\color{lightred}2}}}1
{3}{{{\color{lightred}3}}}1
{4}{{{\color{lightred}4}}}1
{5}{{{\color{lightred}5}}}1
{6}{{{\color{lightred}6}}}1
{7}{{{\color{lightred}7}}}1
{8}{{{\color{lightred}8}}}1
{9}{{{\color{lightred}9}}}1,
basicstyle=\footnotesize\ttfamily, % Standardschrift
numbers=left,               % Ort der Zeilennummern
%numberstyle=\tiny,          % Stil der Zeilennummern
%stepnumber=2,               % Abstand zwischen den Zeilennummern
numbersep=5pt,              % Abstand der Nummern zum Text
tabsize=4,                  % Groesse von Tabs
extendedchars=true,         %
breaklines=true,            % Zeilen werden Umgebrochen
keywordstyle=\color{blue}\bfseries,
frame=b,
commentstyle=\color{brown}\itshape,
stringstyle=\color{editorOcher}\ttfamily, % Farbe der String
showspaces=false,           % Leerzeichen anzeigen ?
showtabs=false,             % Tabs anzeigen ?
xleftmargin=17pt,
framexleftmargin=17pt,
framexrightmargin=5pt,
framexbottommargin=4pt,
%backgroundcolor=\color{lightgray},
showstringspaces=false,      % Leerzeichen in Strings anzeigen ?
}%
%

% https://tex.stackexchange.com/questions/24528/having-problems-with-listings-and-utf-8-can-it-be-fixed
\lstset{style=aistyle,
inputencoding=utf8,
extendedchars=true,
literate=%
{á}{{\'a}}1
{č}{{\v{c}}}1
{ď}{{\v{d}}}1
{é}{{\'e}}1
{ě}{{\v{e}}}1
{í}{{\'{\i}}}1
{ň}{{\v{n}}}1
{ó}{{\'o}}1
{ř}{{\v{r}}}1
{š}{{\v{s}}}1
{ť}{{\v{t}}}1
{ú}{{\'u}}1
{ů}{{\r{u}}}1
{ý}{{\'y}}1
{ž}{{\v{z}}}1
{Á}{{\'A}}1
{Č}{{\v{C}}}1
{Ď}{{\v{D}}}1
{É}{{\'E}}1
{Ě}{{\v{E}}}1
{Í}{{\'I}}1
{Ň}{{\v{N}}}1
{Ó}{{\'O}}1
{Ř}{{\v{R}}}1
{Š}{{\v{S}}}1
{Ť}{{\v{T}}}1
{Ú}{{\'U}}1
{Ů}{{\r{U}}}1
{Ý}{{\'Y}}1
{Ž}{{\v{Z}}}1
}

\DeclareUnicodeCharacter{2212}{\textminus}% requires a unicode capable editor

\addbibresource{text.bib}

\titlecz{Laserový projektor}
\titleen{Laser projector}
\author{Šimon Hrouda}
\field{10}
\school{Gymnázium Brno-Řečkovice, p.~o., Terezy Novákové 2, 621 00 Brno}
\exmentor{Tomáš Rohlínek}
\exmentorstatement{Tomáše Rohlínka}
\inmentor{Mgr. Kateřina Vídenková}
\inmentorstatement{Mgr. Kateřiny Vídenkové}

% Změňte, pokud se liší
%\region{Jihomoravský}
\placefooter{Brno 2024}

\begin{document}
% \newcommand{\bard-gen}[3]{text vygenerován ai #1}
\newcommand{\bardgen}[3]{following text generated by ai (google bard) on #1\\%
  \begin{tcolorbox}[breakable, colback=blue!20]
    #2
  \end{tcolorbox}
  \begin{tcolorbox}[breakable, colback=blue!10, colframe=white]
    #3
  \end{tcolorbox}
}


\maketitle
\fxnote{}

\makecopyrightstatement{V~Brně}

\makethanks{Děkuji svému externímu konzultantovi Tomáši Rohlínkovi a své interní konzultantce Mgr. Kateřině Vídenkové za obětavou pomoc, podnětné připomínky a nekonečnou trpělivost, kterou mi během práce poskytovali.}

\pagestyle{empty}

\section*{Anotace}


\subsection*{Klíčová slova}


\vspace{20mm}

\section*{Annotation}


\subsection*{Keywords}


\newpage
\pagestyle{plain}

\tableofcontents % vysází obsah

%%% Začátek práce
\setcounter{figure}{0}
\setcounter{table}{0}

\newpage

definice pojmů a zkratek
\begin{center}
  \begin{tabular}{c c c}
    CLGS & Closed Loop Galvanometer System & systém galvanometru se zpětnou vazbou \\
    OLGS & Open Loop Galvanometer System   & systém galvanometru bez zpětné vazby  \\
    SPI  & Serial Peripheral Interface     & sériové periferní rozhraní            \\
    DPS & & deska plošných spojů \\
  \end{tabular}
\end{center}

\fxnote{TODO 3. osoba - Práce se zaměřuje,, https://www.sciencedirect.com/search?qs=galvanometer muzu rict, ze jsem neco nezvladl dohledat :)}

% !TeX root = text.tex
\chapter*{Úvod}
\addcontentsline{toc}{chapter}{Úvod} % přidá položku úvod do~obsahu

% \fxnote{?když jsem se~zajímal o~technologii laserových projektoru a~efektu na~diskotekach, zarazilo mě, že jsem nenašel žádnou open source platformu, kterou bych mohl použít a~případně upravovat, kdybych chtěl techcnologii využít, tak jsem se~rozhodl takovou platformu vytvorit}

Laser scanning, technologie rychle se~pohybujícího laserového paprsku, je~využívána v~mnoha oblastech od~laserového promítání, efektů na~diskotékách a~průhledových displejů~(Průhledový displej -- anglicky Head-Up Display (HUD)) v~letadlech, autech či brýlích pro rozšířenou realitu~\cite{laser-huds} přes čtení čárových kódů~\cite{history-of-barcode-scanning} a~3d tisk~\cite{Photo-curing-3D-printing} po~skenování 3D modelů~\cite{3d-model-scan} i~Zemského povrchu~\cite{heightmaps}.

Bohužel ale neexistují žádné uživatelsky přívětivé open-source platformy, kde by~se~s~touto technologií mohli seznámit zájemci o~její rozvíjení.

\subsection*{Cíle}
\addcontentsline{toc}{subsection}{Cíle} % přidá položku úvod do~obsahu
V této práci jsem se~proto rozhodl pro tuto technologii vytvořit vlastní laserový projektor a~naprogramovat pro něj jednoduché uživatelské prostředí.
Toto uživatelské prostředí by~mělo sloužit jako začáteční bod, který zaujme mladé zájemce a~umožní jim si~technologii vyzkoušet.
V případě, že techologie zaujme, mělo by~pro zájemce být jednoduché program pozměnit nebo si~jinak .
% \fxnote{QUESTION\_INTERNAL: jak sepsat cíle? tenhle odstavec tam hezky sedí, ale zvyklejsí jsou asi odrazky, ne?}


\part*{teoretická část}
\fxnote{wtf nechci, aby to  bylo vsechno na nove strance, kdo jsem?}
\addcontentsline{toc}{part}{teoretická část} % přidá položku úvod do obsahu

\input{100-uvod-teor.tex}

% \input{laser_projection.tex}

\chapter{Laser scanning~\cite{scanning-handbook}}
Všechny obrázky v této kapitole pocházejí ze~zdroje~\cite{scanning-handbook}, pokud není uvedeno jinak.

Jako Laser scanning se~označuje technologie využívající rychle pohybující laserový paprsek, tento pohyb je~často zprostředkovaný pohyblivými zrcátky.

Dle stylu pohybu zrcátek se~technologie dají rozdělit na, polygonové skenery, galvanometrové a~MEMS skenery.

\section{Hranolové skenery}
Hranolové skenery se~vyznačují rotujícím hranolem se~zrcadlivými stranami (dále \uv{zrcátky}). Při rotaci hranolu se~mění úhel dopadu laserového paprsku na~zrcátko, a~díky tomu se~mění směr odraženého paprsku, viz. obrázek \ref{fig:polygon-scanner}. \fxnote{adiky carka?}

\begin{figure}[H]
  \centering
  \includegraphics[width=0.5\textwidth]{img/polygon-scanner.jpg}
  \caption{\label{fig:polygon-scanner} mechanika polygonových skenerů}
\end{figure}

% \begin{figure}[!htb]
%   \centering
%   \includegraphics[width=0.5\textwidth]{img/polygon-prismatic-mirror.jpg}
%   \caption{\label{fig:polygon-prismatic-mirror} hranolové zrcátko polygonového skeneru}
% \end{figure}

% \begin{figure}[!htb]
%   \centering
%   \includegraphics[width=0.5\textwidth]{img/polygon-pyramidal-mirror.jpg}
%   \caption{\label{fig:polygon-pyramidal-mirror} pyramidové zrcátko polygonového skeneru}
% \end{figure}
%

S jedním hranolem by~hranolové skenery byly schopny směřovat paprsek pouze v~jedné rovině - při projekci by~bylo možné vykreslit maximálně čáru. Tuto limitaci lze kompenzovat přidáním melého rozdílu ve~směřování každé strany hranolu, viz. obrázek \ref{fig:polygon-angular-variation}. S~touto úpravou každá strana hranolu "vykreslí" jednu, svoji, přímku lehce posunutou vůči přímkách ostatních stran. Hranol s~n-úhelníkovou podstavou je~schopen vykreslit n~přímek.
Další možností je~kombinovat původní pravidelný hranol s~galvanometrem (popsáno níže), kdy galvanometr nastaví jednu souřadnici paprsku a~hranol na~této souřadnici vykreslí přímku.

Tento typ skeneru se~využívá hlavně pro senzory skenující na~přímce (např. skenery čárových kódů~\cite{history-of-barcode-scanning}), nebo při rastrovém procházení plochy (například 3D skenování, nebo promítání ploch, viz. Obrázek \ref{fig:harddrive-projector-youtube}).

\begin{figure}[!htb]
  \centering
  \includegraphics[width=0.8\textwidth]{img/polygon-angular-variation.jpg}
  \caption{\label{fig:polygon-angular-variation} úhlová rozdílnost zrcátek polygonového skeneru a~paprsky od~nich odražené}
\end{figure}


\begin{figure}[!htb]
  \centering
  \includegraphics[width=0.5\textwidth]{img/harddrive-projection.jpg}
  \caption{\label{fig:harddrive-projection} příklad projekce laserového projektoru s~polygonovým skenerem; zdroj~\cite{harddrive-projector-youtube}}
\end{figure}

\section{Galvanometrové skenery}
V galvanometrových skenerech paprsek odráží zrcátka/o připevněná/o na~páru galvanometrů.

\subsection{Galvanometr}
Slovem galvanometr se~označuje přístroj úrčený k~detekci nebo měření velice malého elektrického proudu~\cite{galvo-definition}. Galvanometry při měření využívají interakce magnetického pole trvalého magnetu a~cívky protéké proudem. Tato interakce vychýlí ručičku ukazující na~stupnici, nebo zrcátko odrážející paprsek, který dopadá na~stupnici.~\cite{wiki-galvo}

Galvanometry se~dají rozdělit na~galvanometry bez zpětné vazby (open-loop) a~se zpětnou vazbou. K~těm bývají připojeny ovládací obvody, které z~galvanometrů získávají informace o~jejich pohybu a~podle nich regulují signál posílaný do~galvanometrů.~\cite{wiki-galvo}

Dále se dělí dle pohyblivé součástky. V galvanometru je buď trvalý magnet pevně ukotven a cívka pohyblivá (moving coil), nebo naopak (moving magnet). % https://prirucka.ujc.cas.cz/?id=151#nadpis3

Dnes se~v~kontextu laserových skenerů prakticky vždy používají galvanometry s~pohyblivým magnetem a se~zpětnou vazbou. Ta je zajištěna čtením z variabilního kondenzátoru umístěného v galvanometru.

\begin{figure}[!htb]
  \centering
  \includegraphics[width=1\textwidth]{img/galvanometer-detail.jpg}
  \caption{\label{fig:galvanometer-detail} zapojení a vnitřní konstrukce galvanometrů}
\end{figure}

\subsection{Konstrukce galvanometrových skenerů\label{sec:galvo-scanner-construction}}
Jeden galvanometrový skener vždy ovládá jednu osu pohybu paprsku, buď X~nebo Y.

Narozdíl od~hranolových skenerů je~s galvanometrovým skenerem možné zastavit obě osy pohybu - vykreslovat na~sebe kolmé čáry.

\begin{figure}[!htb]
  \centering
  \includegraphics[width=1\textwidth]{img/scanner-constructions.jpg}
  \caption{\label{fig:scanner-constructions} různé konstrukce galvanometrových skenerů}
\end{figure}

\section{Má volba skeneru}
\fxnote{sekce mozna patri do~prakticke}
ja vyuzivam galva, protoze se~s nima da~nejlip pohrat, jsou nejuniverzalnejsi a~tim padem nejvic zaujmou - cil

% \input{galvanometr.tex}

% \input{.tex}


\part*{praktická část}
\fxnote{wtf nechci, aby to  bylo vsechno na nove strance, kdo jsem?}
\addcontentsline{toc}{part}{praktická část} % přidá položku úvod do obsahu

\input{200-uvod-prak.tex}

% !TeX root = text.tex
\chapter{hardware}

\section{Raspberry Pi}
\fxnote{TODO: rpi specs}

\section{Set galvanometrů se~zrcátky} \label{sec:my-galvos}
\subsection{Výběr skeneru}
Pro tuto práci byl vybrán galvanometrový skener, protože je~nejdostupnější a~protože potenciálním uživatelům nejlépe představí technologii.

Oproti hranolovým skenerům jim tožiž dává více možností, jak s paprskem pohybovat.
Můžou se~rozhodnout, že jej využijí jako hranolový skener, pokud nahrají soubor procházející promítací plochu po~řádcích.

Oproti dalším typům skenerů je~názornější, ostatní typy skenerů jsou totiž příliš malé a~není na~nich vidět princip funkce nebo je~jejich fungování nadmíru abstraktní a~těžko pochopitelné.

\subsection{Zapojení galvanometrového setu}
Samotné galvanometry jsou zapojeny do~řídící desky, která s nimi byla zakoupena, ta je vidět na obrázku~\ref{fig:hw_galvoboard}.

Řídící deska požaduje symetrický zdroj napětí 15~V, tzn. $+15$~V a~$-15$~V a~samozřejmě připojení k zemi. Také přijímá dva bipolární diferenciální analogové signály s~rozsahem diferenciálního napětí $-10$~V až $+10$~V. Každý signál udává vychýlení jednoho ze~dvou galvanometrů, což obvykle znamená výslednou pozici laserového paprsku v~osách X~a~Y.

\begin{figure}[htb]
  \centering
  \includegraphics[width=1\textwidth]{img/hw_galvoboard.jpg}
  \caption{\label{fig:hw_galvoboard} Řídící deska galvanometrů s vyznačenými konektory a hřejícími čipy}
\end{figure}

\subsection{bipolární diferenciální analogový signál~\cite{ilda-signal-spec}}
Diferenciální signál je~signál přenášený dvěma vodiči, každý z~nich přenáší stejný signál, jen s~opačnou polaritou. Kontakt označený $(+)$ je~považován za~nosič základního signálu, zatímco kontakt označený $(-)$ je~považován za~nosič invertovaného signálu. Výsledné diferenciální napětí je~napětí na~základním nosiči vůči napětí na~obráceném nosiči, tzn.~$V_{dif} = V_{(+)} - V_{(-)}$

Bipolární signál znamená, že na~napětí každém z~kontaktů $(+)$ a~$(-)$ může dosahovat kladných i~záporných hodnot.

Tudíž cheme-li disáhnout diferenciálního napětí $+10~V$, musí mít základní signál napětí $+5~V$ a~obrácený signál $-5~V$. Záporné diferenciální napětí bude ve~chvíli, kdy je~napětí základního signálu záporné a~napětí obráceného signálu kladné.

\subsection{Zahřívání čipů řídící desky galvanometrů} \label{sec:galvoboard-chips-heating-up}
Dva z~čipů na~řídící desce při chodu systému výrazně zahřívájí. Na~tyto čipy naštěstí už od~výroby desky je~připevněna malá hliníková destička. Ta~má sloužit jako chladič, ale i~s ní se~čipy v~otevřeném prostoru zahřívají na~teploty blízké 60~\degree{}C.
Dva zmíněné čipy jsou čipy TDA2030A od~firmy STMicroelectronics. Ty~by~měly dle datasheetu vydržet až 150~\degree{}C, ale dá se~předpokládat, že v~uzavřeném pouzdru budou čipy dosahovat vyšších teplot, než v~otevřeném prostoru. I~kdyby nedosáhly pro sebe kritických 150~\degree{}C, rozhodně není žádoucí, aby uvnitř projektoru desky dosahovaly vysokých teplot.

I proto byl do~projektoru zabudován chladič, Více o~způsobu jeho připevnění a~distribuci chlazení mezi ostatní komponenty se~dočtete v~kapitole~\ref{sec:krabick-design-priorities}.


\section{laser}
Jako zdroj laserového paprsku byl využit RGB laserový modul, skládající se ze tří barevných diod a dichroických zrcátek.

\subsection{Dichroická zrcadla~\cite{dichronic-mirrors}}

\fxnote{TODO: slouzi k}

Dichroická zrcadla jsou zrcadla s výrazně rodílnými odrazovými nebo průchodovými vlastnostmi pro dvě různé vlnové délky odraženého~/~procházejícího světla.

Většina dichroických zrcadel jsou dielektrická zrcádla \footnote{Dielektrická zrcadla jsou zrcadla, skládající se z mnoha tenkých vrstev různě opticky propustných materiálů.}, existují ale také krystalická zrcadla\footnote{Krystalická zrcadla jsou zrcadla, jejiž odrážlivá vrstva se skládá z monokrystalického materiálu, typicky polovodiče.}.


\section{LCD displej}
\section{rotační enkodér}

\section{HAT}
Pro ovládání výše popsaného hardwaru je~zapotřebí několik specifických obvodů.
Kvůli jejich specifičnosti tyto obvody nejsou volně dostupné k~zakoupení na~předem vytvořených destičkách. Proto bylo zapotřebí je~z jednotlivých součástek vyrobit na~míru.

Obvody byly navrženy v~programu KiCad...\fxnote{bud spojit vety, nebo k~prvni neco jeste dopsat}
Následně pro ně v~tomtéž programu byla nadesignována deska plošných spojů. Na~této desce se~vyskytují obvody \fxnote{todo dac+amps, bat\_probe, -15V}.
Kromě nich byly na~desku přidány konektory k~jednotlivým barevným vstupům laseru, LCD displeji a~k rotačnímu enkodéru, které jsou přímo napojeny na~40 pinový GPIO konektor Raspberry Pi.
Deska byla designována jako tzv. HAT, to~znamená, že sama na~tomto konektoru drží a~nezabírá o~moc víc místa, než samotné Raspberry Pi.
\fxnote{TODO: obrazek desky (maybe mounted)}

\subsection{Zdroj $-15$~V}

\subsection{obvod pro generování analogového signálu}
Jak popsáno v~sekci \ref{sec:my-galvos}, řídící deska galvanometrů přijmá dva bipolární diferenciální analogové signály v~rozpětí $-5$~V až $+5$~V.

Obvod, který se~stará o~vytváření tohoto signálu je~založený na~obvodu ze~zdroje~\cite{lasershow-with-real-galvos}.
Vytváření tohoto signálu je~rozděleno do~dvou částí. Nejdříve DAC (digital-to-analog converter, D/A převodník) připojený k~RPi vytvoří signál v~rozpětí 0 až 5~V a~následně je~tento signál pomocí operačního zesilovače převeden na~požadované rozpětí, tj. $-15$~V až $+15$~V.
Jednotlivé části tohoto obvodu jsou blíže popsány v~následujících kapitolách. Celé zapojení je~vidět na~obrázku \ref{fig:dac_board}.
\fxnote{unreadable text, make schem more compact}
\begin{figure}[!htb]
  \centering
  \includegraphics[width=1\textwidth]{img/dac_board.png} 
  \caption{\label{fig:dac_board}Zapojení DAC a~zesilovačů k~RPi a~řídící desce galvanometrů}
\end{figure}

\subsubsection{dac\cite{mcp4822-dsh}}
K generování signálu v~rozpětí 0--5~V byl využit dvoukanálový D/A převodník\footnote{obvod, který na~základě instrukcí přijatých digitálně generuje analogové napětí} MCP4822.
Tento čip podporuje komunikaci přes rozhraní SPI, pracuje s~napájecím napětím 5~V a~s 12bitovým rozlišením (je schopen vygenerovat 4~096 různých napětí) na~dvou kanálech.

RPi komunikuje s~čipem pomocí rozhraním SPI.
\fxnote{TODO more spec}
Tato knihovna poskytuje následující funkce, se~kterými pracuji v~mém kódu.
\begin{itemize}
\item
\lstinline[language=C]!bool mcp4822_initialize();!
\item
\lstinline[language=C]!bool mcp4822_set_voltage(mcp4822_channel_t channel, uint16_t value_mV);!
\item
\lstinline[language=C]!void mcp4822_deinitialize();!
\end{itemize}
\subsubsection{amps\cite{tl082-dsh}}
K modifikaci signálu z~DAC na~bipolární diferenciální analogový signál slouží pro každý kanál jeden čip TL082, který obsahují dva operační zesilovače. Ty~jsou zapojeny dle schématu na~obrázku \ref{fig:ilda_amps-scheme}.

Signál první operační zesilovač zesílí a~posune dle nastavení potenciometrů Ygain(zesílení) a~Yoffset(posun) a~zároveň invertuje. Tento invertovaný signál následně druhý operační zesilovač opět invertuje, získav základní signál pro řídící desku galvanometrů.

\begin{figure}[!htb]
  \centering
  \includegraphics[width=1\textwidth]{img/ilda_amps.png} 
  \caption{\label{fig:ilda_amps-scheme} Zapojení čipu TL082 pro jeden kanál řídící desky galvanometrů}
\end{figure}

\fxnote{TODO more spec}
Tyto čipy mi~napěťové rozpětí zvýší z~0--5~V na~$-15$~V až $+15$~V.

zesilovac - cteni baterek \url{https://is.muni.cz/el/sci/jaro2017/F5090/um/E17_P8.pdf}


\fxnote{TODO cos udelal svyho vlastne a jak to facha}

\section{cooler}

\section{napájení}
\fxnote{TODO ay tak co, zvladls to dat na baterky?}


% !TeX root = text.tex
\chapter{Software}

Tato kapitola se~zabývá softwarovou výbavou laserového projektoru. Detailně popisuje klíčové programy, jejich funkce a~způsob, jakým mezi sebou komunikují.
Mezi tyto programy patří program lasershow, který obsluhuje vykreslování, programy pro interakci s~uživatelem (dále \uv{frontendové programy}) jako UI, web\_ui a~discord\_bot, a~také program wifi\_manager, který spravuje wifi připojení Raspberry Pi.
Programy wifi\_manager a discord\_bot se dají označit za backendové programy, protože neinteragují s uživatelem. V neposlední řadě popisuje také instalační skript, který výrazně zjednodušuje instalaci všech zmíněných programů a~jejich závislostí.

% O~řízení galvanometrů se~stará program lasershow, který je~psaný v~jazyce c++ pro maximální rychlost. Tento program běží na~pozadí a~čeká na~příkazy od~programů určených k~interakci s~uživatelem. Na~tento program se~zaměřuje kapitola lasershow. \fxnote{TODO: odkaz}

% Dále jsou tu~programy, které se~starají o~interakci s~uživatelem. Tyto programy přijímají příkazy od~uživatele a~posílají je~programu lasershow. Navíc od~lasershow získávají výstup, který následně zprostředkovávají uživateli; důkladněji popsáno v~kapitole~\ref{sec:comms}.
% Mezi tyto programy patří programy UI, web\_ui a~discord\_bot. Program UI~spravuje OLED displej, přijímá od~uživatele vstup rotačním enkodérem a~je~psaný v~c++ pro jednodušší interakci s~hardwarem. Program web\_ui využívá runtime Node.js, ve~kterém je~nenáročné vytvořit http server dostupný z~lokální sítě. \fxnote{(A)?} Program discord\_bot, také využívající Node.js, přijímá příkazy z~chatovací aplikace discord a~je~přístupný i~přes internet.

% Nakonec je~tu~program wifi\_manager, ten spravuje wifi připojení RPi, je~psaný v~Node.js a~komunikuje s~programy, které interagují s~uživatelem stejně jako program lasershow.

\section{komunikace mezi programy} \label{sec:comms}
Všechny tyto programy jsou propojeny síťovými sockety zprostředkovanými knihovnou ZeroMQ, která nabízí frontu\footnote{Ve frontě jsou zprávy seřazeny od~té nejdříve odeslané.} zpráv, bez potřeby samostatně běžícího brokeru.

Tato knihovna je~využita k~vytvoření dvou socketů, jedním lasershow přijímá příkazy od~uživatele prostřednictvím ostatních programů (vstupní socket na~portu 5557, viz obr. \ref{fig:tcp5557}) a~do druhého posílá informace ostatním programům (výstupní socket na~portu 5556, viz obr. \ref{fig:tcp5556}), aby je~zprostředkovaly uživateli. Do~prvního zmíněného posílají progamy interagující s~uživatelem příkazy pro programy lasershow a~wifi\_manager. Do~druhého posílá lasershow informace o~stavu a~změnách nastavení  a~také wifi\_manager informace o~stavu a~změnách v~nastavení WiFi.

Příkazy pro programy lasershow a~wifi\_manager vypadají následovně
\fxnote{TODO: příklady příkazů pro lasershow a~wifi\_manager z~\url{https://github.com/phuid/laser_projector/blob/master/README.md}}
\fxnote{TODO: příklady status infos od~lasershow a~wifi\_manager z~\url{https://github.com/phuid/laser_projector/blob/master/README.md}}

\begin{figure}[!htb]
  \centering
  \includegraphics[width=0.5\textwidth]{img/tcp5557.png}
  \caption{\label{fig:tcp5557}komunikace mezi programy vstupním socketem na~portu 5557}
\end{figure}
\begin{figure}[!htb]
  \centering
  \includegraphics[width=0.5\textwidth]{img/tcp5556.png}
  \caption{\label{fig:tcp5556}komunikace mezi programy výstupním socketem na~portu 5556}
\end{figure}

\section{lasershow}

Program lasershow je psaný v jazyce c++, který je kompilovaný a obecně považovaný za jeden z nejrychlejších jazyků. Druhé zmíněné se hodí, jelikož chceme vykreslovat co možná nejrychleji.

Tento program zaregistruje vstupní TCP socket na portu 5557 a knihovnou ZeroMQ se na něm přihlásí k odběru zpráv, které do něj publikují ostatní programy. Zárověn podobně zaregistruje výstupní socket na portu 5556, do kterého později bude posílat zprávy pro programy, které interagují s uživatelem.

Následně se připojí k DAC a čeká na zprávy od ostatních programů. Jakmile zprávu obdrží, zpracuje ji a pokud je požadována změna nastavení, okamžitě ji provede a aktuální nastavení si uloží do souboru, jestliže je požadováno vykreslení obrazu ze souboru, začne obraz vykreslovat. Při tom průběžně posílá informace o stavu vykreslování do výstupního socketu. I při vykreslování obrazu tento program zpracovává zprávy a pokyny ze vstupního socketu.

Program byl původně převzat z projektu \url{https://github.com/tteskac/rpi-lasershow}\footnote{staženo 28.~12.~2023}, následně byl ale přepsán skoro ve všech ohledech a z původního programu zbylo asi 20 řádků.
\fxnote{TODO: odkud jsem to vzal a prepsal a jak moc jsem toho udelal a s jakymy vysledky}

\fxnote{TODO: diagram programu}

\fxnote{TODO: priklad zmq}\

\lstinputlisting[language=c++, style=code]{code_examples/zmq_server.cpp}
\lstinputlisting[language=c++, style=code]{code_examples/zmq_client.cpp}


\section{wifi\_manager}

V rámci této práce byl vyvinut ještě jeden program, který se přímo nepodílí ani na projekci, ani na interakci s uživatelem.

Program wifi\_manager je také napsaný v jazyce JavaScript s využitím runtime Node.js. Registruje se ke stejným socketům jako lasershow, přijímá příkazy týkající se nastavení WiFi na Raspberry Pi TCP socketem na portu 5557 a odesílá zpětnou vazbu na TCP socket s portem 5556.

\fxnote{TODO: jak se komunikace s lasershow odlisuje od wifi\_managera}

\fxnote{TODO: ukazka(idk what)}

Hlavním úkolem tohoto programu je správa a konfigurace WiFi připojení na Raspberry Pi. Přijímá příkazy od ostatních programů a nastavuje WiFi parametry na základě těchto příkazů. Tím umožňuje uživatelům snadno a pohodlně nastavit WiFi připojení na svém zařízení.

Stejně jako lasershow, wifi\_manager také posílá zpětnou vazbu ostatním programům, aby informoval o stavu a změnách v nastavení WiFi. Tímto způsobem je zajištěna komunikace a synchronizace mezi všemi programy v laserovém projektoru.

Celkově wifi\_manager přispívá k plynulému a efektivnímu provozu laserového projektoru tím, že umožňuje snadnou správu a konfiguraci WiFi připojení na Raspberry Pi.

\section{UI}

Program UI~je také psaný v~jazyce c++ a~využívá knihovnu WiringPi, která umožňuje jednoduchou komunikaci s~GPIO piny Raspberry Pi. Tento program ovládá OLED displej, který je~připojený na~Raspberry Pi~pomocí rozhraní I2C, a~přijímá vstup od~uživatele čtením rotačního enkodéru s~tlačítkem.

Program se~při začátku exekuce pomocí knihovny ZeroMQ přihlásí ke~vstupnímu socketu a~k odběru zpráv z~výstupního TCP socketu, kam publikuje zprávy o~stavu vykreslování program lasershow. Dále si~pomocí knihovny wiringPi zaregistruje zpracovávání přerušení z~enkodéru a~tlačítka na~něm a~čeká buď na~interakci s~uživatelem, který by~skrz něj poslal zprávy programu lasershow, nebo na~zprávy od~lasershow, které by~zobrazil uživateli.

\fxnote{TODO: diagram programu}

\section{web\_ui}

Narozdíl od~předchozích dvou zmiňovaných programů je~program web\_ui psaný v~jazyce javascript, ten nepatří mezi nejrychlejší, ale díky runtime Node.js a~knihovnám http a~formidable v~něm bylo časově nenáročné vytořit http web server.

Tento server běží na~portu 3000 a~je dostupný z~lokální sítě (tzn. přímo z\~Raspberry Pi~na adrese http://localhost:3000 nebo z~jakéhokoliv zařízení na~stejné lokální síti na~ip adrese RPi).
Program je~využíván pro jednoduchou interakci s~uživatelem, který může pomocí webového prohlížeče ovládat laserový projektor pár kliknutími i~zadávat vlastní příkazy klávesnicí.

\fxnote{na webu jsou konzole pro ssh, wifiman a~lasershow, taky fast project forms}

\fxnote{TODO: příklad http serveru}
\lstinputlisting[language=JavaScript, style=code]{code_examples/http_static_files.js}

Stejně jako program UI~za pomoci knihovny ZeroMQ tento program odebírá z~výstupního socketu zprávy o~průběhu vykreslování od~programu lasershow a~odesílá mu~pokyny uživatele na~vstupní socket.

\fxnote{TODO: příklad přihlášení k~socketům v~js}

\fxnote{TODO: xterm + ssh}

\section{discord bot}

Posledním programem, který je~využíván k~interakci s~uživatelem je~discord\_bot, který je~také psaný v~jazyce javascript v~runtime Node.js, stejně jako předchozí programy se~přihlásí k~socketům knihovnou zmq, ale na~rozdíl od~nich tento program může interagovat s~uživatelem přes internet ať už je~kdekoliv na~světě.
Pomocí knihovny discord.js se~přihlásí k~předem vytvořenému bot účtu, který může na~předem vytvořeném discord serveru čekat na~zprávy od~uživatele, ty~posílat do~vstupního socketu a~posílat uživateli zpětnou vazbu, kterou příjme z~výstupního socketu.



\section{install.sh}
\fxnote{TODO: install.sh}


\chapter{Diskuze}

Existuje nemalé množství open-source implementací laserových projektorů.

Většina  z~nich  ale~není uživatelsky přívětivá, to~neznamená, že jsou tyto projekty špatné,  ale~hodí se~uživatelům, kteří jsou pokročilejší programátoři. 
Jednou  z~nich  byl~i tento projekt inspirován.
Ta se~jmenuje rpi-lasershow  a~je~dostupná z~\cite{rpi-lasershow}. Obsahuje základní vykreslovací program, který umí číst soubory formátu .ild. Její hlavní problém spočívá  v~neexistujícím uživatelském prostředí.

Nad ostatní implementace vystupuje projekt openlase~\cite{openlase}.  Ten~je~koncipován jako knihovna, kterou zájemci mohou začlenit do~vlastních projektů.
Hardware  k~němu navžený  ale~vyžaduje připojení  k~počítači  i~zásuvce, což by~mohlo být nepraktické, kdyby se~hodilo projekci předvést potenciálním zájemcům  o~technologii naživo.

Grafické uživatelské prostředí obsahuje projektor vypracovaný  v~diplomové práci Bc. Pavla Svobody~\cite{vut-chabr}.  Ten~je~ale, stejně jako openlase, konstruován  s~připojením  k~počítači  a~zásuvce.

Za zmínku také stojí projekt zpracovaný  v~youtube videu kanálu \uv{Ben Makes Everything}~\cite{harddrive-projector-youtube}, který obsahuje  i~aplikaci na~mobilní telefon. Ta ~s~projektorem komunikuje přes bluetooth. Tento projektor  ale~uživatelům umožňuje vypsat  jen~text, žádné vlastní obrázky  ani~videa.
Zajímavé na~tomto projektu je, že je~založen na~hranolovém skeneru, který tvůrce vytvořil ze~starého hard-disku. Je~tedy poměrně dostupný  pro~kohokoliv  s~přístupem ke~3d tiskárně, která byla při výrobě projektoru použita.
Představení hranolových skenerů je~užitečné vzhledem  k~jejich častému využití  v~aplikacích jiných, než je~vykreslování obrazu. Grafiky vykreselné galvanometrovými skenery můžou být zajímavější, než ty~vykreslené skenery hranolovými.

Když opustíme sféru open-source, samozřejmě existuje spousta komerčních řešen. Například od~firem Pangolin nebo Laserworld. Ty~jsou často výkonnější, než výše popsané projekty  a~jsou využívané  pro~vytváření profesionálních laserových efektů.

Velice zajímavým zařízením je~Laser Cube~\cite{lasercube} od~firmy Wicked Lasers. Je ~to~komerční laserový projektor  pro~širokou veřejnost  s~velice jednoduchým uživatelským prostředím. Open-source řešení včetně této práce by ~se~k němu daly přirovnat jako alternativy  pro~technicky zdatné jedince, kteří jsou schopni  s~technologií pracovat sami.


\newpage
\chapter*{Závěr}

V rámci této práce byl navrhnut a sestrojen funkční RGB laserový projektor, kromě jednoho vadného obvodu, kvůli kterému musí projektor zatím být připojen k zásuvce.

K tomuto projektoru byla naprogramována softwarová výbava umožňující promítat barevné obrazce a skrze uživatelské prostředí ovládat nastavení promítání přímo na zařízení, z webového prohlížeče, nebo přes chattovací aplikaci discord odkudkoliv na světě. Tímto uživatelským prostředím je také možné nastavovat WiFi připojení projektoru.

Celý projekt byl vystaven na webovou platformu pro sdílení open-source projektů \url{github.com}, kde ho může najít kdokoliv.

% \addcontentsline{toc}{chapter}{Závěr} \fxnote{FIXME proc vsichni maji zaver v~obsahu jako section, kdyz pak~vypada, ze~je~pod~posledni kapitolou??}
% \fxnote{TODO závěr}\fxnote{TODO muj~projekt je~dostupný na~githubu}
% \fxnote{vytvoril jsem projektor, ten~je~mozne sestavit s~minimální znalostí elektroniky, staci umet pajet, jde~mi~o~to~zajemcum predstavit jak~to~programovat, jak~funguji vektorove formaty, ne~jak~funguje hw, ten~se~totiz meni - MEMS}
% \fxnote{tim ze~mam~linux lasershow obcas zpomali, obcas zrychli, samozrejme mam~osetreny, aby~nezrychlily framy, ale~pointy ano}

\newpage
\printbibliography[title=Literatura]
\addcontentsline{toc}{section}{Literatura}

\listoffigures
\addcontentsline{toc}{section}{Seznam obrázků}

\listoftables
\addcontentsline{toc}{section}{Seznam tabulek}
%
% \listoflistedequation
% \addcontentsline{toc}{section}{Seznam rovnic}

\end{document}
