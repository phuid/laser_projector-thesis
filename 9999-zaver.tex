\newpage
\chapter*{Závěr}

V rámci této práce byl navrhnut a sestrojen funkční RGB laserový projektor, kromě jednoho vadného obvodu, kvůli kterému musí projektor zatím být připojen k zásuvce.

K tomuto projektoru byla naprogramována softwarová výbava umožňující promítat barevné obrazce a skrze uživatelské prostředí ovládat nastavení promítání přímo na zařízení, z webového prohlížeče, nebo přes chattovací aplikaci discord odkudkoliv na světě. Tímto uživatelským prostředím je také možné nastavovat WiFi připojení projektoru.

Celý projekt byl vystaven na webovou platformu pro sdílení open-source projektů \url{github.com}, kde ho může najít kdokoliv.

% \addcontentsline{toc}{chapter}{Závěr} \fxnote{FIXME proc vsichni maji zaver v~obsahu jako section, kdyz pak~vypada, ze~je~pod~posledni kapitolou??}
% \fxnote{TODO závěr}\fxnote{TODO muj~projekt je~dostupný na~githubu}
% \fxnote{vytvoril jsem projektor, ten~je~mozne sestavit s~minimální znalostí elektroniky, staci umet pajet, jde~mi~o~to~zajemcum predstavit jak~to~programovat, jak~funguji vektorove formaty, ne~jak~funguje hw, ten~se~totiz meni - MEMS}
% \fxnote{tim ze~mam~linux lasershow obcas zpomali, obcas zrychli, samozrejme mam~osetreny, aby~nezrychlily framy, ale~pointy ano}