% !TeX root = text.tex
\chapter*{Úvod}
\addcontentsline{toc}{chapter}{Úvod} % přidá položku úvod do~obsahu

\fxnote{?když jsem se~zajímal o~technologii laserových projektoru a~efektu na~diskotekach, zarazilo mě, že jsem nenašel žádnou open source platformu, kterou bych mohl použít a~případně upravovat, kdybych chtěl techcnologii využít, tak jsem se~rozhodl takovou platformu vytvorit}

Laser scanning, technologie rychle se~pohybujícího laserového paprsku, je~využívána v~mnoha oblastech od~laserového promítání, efektů na~diskotékách a~Heads Up~Displejů v~letadlech či autech~\cite{huds-in-driving} přes čtení čárových kódů~\cite{history-of-barcode-scanning} a~3d tisk~\cite{Photo-curing-3D-printing} po~skenování 3D modelů~\cite{3d-model-scan} i~Zemského povrchu~\cite{heightmaps}.

Bohužel ale neexistují žádné uživatelsky přívětivé open-source platformy, kde by~se s~touto technologií mohli seznámit zájemci o~její rozvíjení.

\subsection*{Cíle}
\addcontentsline{toc}{subsection}{Cíle} % přidá položku úvod do~obsahu
V této práci jsem se~proto rozhodl pro tuto technologii vytvořit vlastní laserový projektor a~naprogramovat pro něj jednoduché uživatelské prostředí.
Toto uživatelské prostředí by~mělo sloužit jako začáteční bod, který zaujme mladé zájemce a~umožní jim si~technologii vyzkoušet.
V případě, že techologie zaujme, mělo by~pro zájemce být jednoduché program pozměnit nebo si~jinak .
\fxnote{QUESTION\_INTERNAL: jak sepsat cíle? tenhle odstavec tam hezky sedí, ale zvyklejsí jsou asi odrazky, ne?}
