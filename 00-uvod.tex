% !TeX root = text.tex
\chapter*{Úvod}
\addcontentsline{toc}{chapter}{Úvod} % přidá položku úvod do~obsahu

\fxnote{preferuj itemize nad čarkou oddělenýma seznamama}

% \fxnote{?když jsem se~zajímal o~technologii laserových projektoru a~efektu na~diskotekach, zarazilo mě, že jsem nenašel žádnou open source platformu, kterou bych mohl použít a~případně upravovat, kdybych chtěl techcnologii využít, tak~jsem se~rozhodl takovou platformu vytvorit}

Laser scanning, technologie rychle se~pohybujícího laserového paprsku, je~využívána v~mnoha oblastech od~laserového promítání, efektů na~diskotékách~\cite{laser-projection} a~průhledových displejů~(Průhledový displej -- anglicky Head-Up Display (HUD)) v~letadlech, autech či brýlích pro~rozšířenou realitu~\cite{laser-huds}, přes čtení čárových kódů~\cite{history-of-barcode-scanning} a~3D tisk~\cite{Photo-curing-3D-printing}, po~skenování 3D modelů~\cite{3d-model-scan} i~zemského povrchu~\cite{heightmaps}.

Bohužel ale~neexistují žádné uživatelsky přívětivé open-source platformy, kde~by~se~s~touto technologií mohli seznámit zájemci o~její rozvíjení.

\chapter*{Cíle}
Cílem práce je~vytvořit pro~technologii laser scanningu jednoduché uživatelské prostředí a~následně toto prostředí spolu s~návodem, jak~svůj projektor zprovoznit, vystavit na~server \url{github.com}, kde~ho~případní zájemci najdou.
Toto uživatelské prostředí by~mělo sloužit jako začáteční bod, který zaujme mladé zájemce a~umožní jim~si~technologii vyzkoušet.
V případě, že zájemce techologie zaujme, mělo by~pro~ně být jednoduché program pozměnit nebo si~jinak přizpůsobit chování projektoru.
