\section{Displej z~tekutých krystalů (LCD)}
Pro zobrazování informací uživateli přímo na~zařízení  byl~využit alfanumerický\footnote{Alfanumerický -- Řídící jednotka displeji místo pixelů posílá celé znaky, které sám vykresluje.}  LCD~s řadičem HD44780 a ~s~rozlišením 20~x~4 znaky.  K~displeji je~také připojen I$^{2}$C převodník, který slouží jako prostředník mezi řadičem displeje a~Raspberry Pi.
Komunikační protokol  LCD~totiž využívá podstatně více kontaktů, než I$^{2}$C sběrnice, kterou Raspberry Pi~komunikuje  s~převodníkem.
% Převodník je ~k~pinům Raspberry Pi~připojen  dle~tabulky~\ref{tab:LCD_conn}.

\begin{figure}[htb]
  \centering
  \begin{minipage}{0.45\textwidth}
    \centering
    \includegraphics[width=1\textwidth]{img/LCD_front.jpg} % first figure itself
    \caption{\label{fig:LCD_front} Displej z~tekutých krystalů (LCD); Převzato a~upraveno z~\cite{laskakit-LCD}}
  \end{minipage}\hfill
  \begin{minipage}{0.45\textwidth}
    \centering
    \includegraphics[width=1\textwidth]{img/LCD_back.jpg} % second figure itself
    \caption{\label{fig:LCD_back} I$^{2}$C převodník napojený na~LCD~\cite{laskakit-LCD}}
  \end{minipage}
\end{figure}

% \begin{table}[htb]
%   \centering
%   \begin{tabular}{c|c}
%     kontakt převodníku & kontakt  RPi~\\
%     \hline
%      GND~               &  GND~        \\
%     5V                 & 5V          \\
%      SCK~               & GPIO3       \\
%      SDA~               & GPIO2       \\
%      LED~               & GPIO18      \\
%   \end{tabular}
%   \caption{\label{tab:LCD_conn} Připojení kontaktů I$^{2}$C převodníku na~kontakty RPi.}
% \end{table}


\section{Rotační enkodér~\cite{how-encoders-work}\cite{rotary-encoder-cvut}}
Rotační enkodér je ~typ~pozičního senzoru používaný  k~měření rotace otáčivé hřídele. Existuje mnoho druhů enkodérů, rozdělují se ~dle~signálu, který vydávají a ~dle~technologie, kterou měří rotaci hřídele.  V~této práci je~použit mechanický inkrementální enkodér  s~tlačítkem.

Na obrázku~\ref{fig:encoder-working} je~vidět,  jak~enkodér funguje uvnitř.  Dva~kontakty A ~a~B při rotaci získávají a~ztrácí kontakt  s~kontaktem C. Připojíme-li ke~kontaktu  C~zem a~ke~kontaktům A ~a~B pull-up rezistory (klidně softwarově), dá se~toto získávání a~ztrácení kontaktu zaznamenat do~grafu na~obrázku~\ref{fig:encoder-graph} jako  dva~signály obdélníkového průběhu vzájemně fázově posunuté  o~90 stupňů.

\begin{figure}[htb]
  \centering
  \begin{minipage}{0.45\textwidth}
    \centering
    \includegraphics[width=1\textwidth]{img/encoder-working.jpg}
    \caption{\label{fig:encoder-working} Vnitřní schéma enkodéru~\cite{how-encoders-work}}
  \end{minipage}\hfill
  \begin{minipage}{0.45\textwidth}
    \centering
    \includegraphics[width=1\textwidth]{img/encoder-graph.jpg}
    \caption{\label{fig:encoder-graph} Výstup enkodéru~\cite{how-encoders-work}}
  \end{minipage}
\end{figure}

Použitý rotační enkodér má další  dva~kontakty připojené  k~tlačítku  pod~rotující hřídelí.
Ke~čtení stisknutí tlačítka je~potřeba připojit jeden kontakt  k~zemi, tedy ke~kontaktu  C~a~druhý kontakt na~pull-up rezistor (klidně softwarově).
Celé zapojení enkodéru je~naznačeno na~obrázku~\ref{fig:encoder-pinout}. Kontakty tlačítka jsou  v~něm označeny S1 a~S2.
% Enkodér je ~k~RPi připojen  dle~tabulky~\ref{tab:enc_conn}.

% \begin{table}[htb]
%   \centering
%   \begin{tabular}{c | c}
%     kontakt enkodéru & kontakt  RPi~\\
%     \hline
%      C~               &  GND~        \\
%     S2               &  GND~        \\
%     A~               & GPIO5       \\
%      SDA~             & GPIO6       \\
%     S1               & GPIO13      \\
%   \end{tabular}
%   \caption{\label{tab:enc_conn} Připojení kontaktů rotačního enkodéru na~kontakty RPi.}
% \end{table}


\subsection{Čtení pozice z~rotačního enkodéru}
U rotačního enkodéru se~při každé otáčce změní připojení pinů několikkrát. Chceme-li pozorovat pouze počet těchto změn, stačí spočítat změny na~jednom kontaktu. Pokud je ~ale~zapotřebí pozorovat  i~směr otáčení, je~nutné pozorovat stav obou kontaktů. Pokud se~enkodér otáčí po~směru hodinových ručiček, kontakt A~bude fázově posunut  o~90 stupňů napřed oproti kontaktu B.
Pokud se~eknodér otáčí proti směru hodinových ručiček, bude naopak kontakt  B~o 90 stupňů napřed oproti kontaktu A. Časový průběh stavu kontaktů je~naznačen na~obrázku~\ref{fig:encoder_data}.
\begin{figure}[htb]
  \centering
  \includegraphics[width=0.5\textwidth]{img/encoder-pinout.jpg}
  \caption{\label{fig:encoder-pinout} Schéma zapojení rotačního enkodéru.}
\end{figure}
\begin{figure}[htb]
  \centering
  \includegraphics[width=1\textwidth]{img/encoder_data.jpg}
  \centering
  \caption{\label{fig:encoder_data} Časový průběh stavu kontaktů rotačního enkodéru při otáčení hřídelí na~obě strany}
\end{figure}
