% !TeX root = text.tex
\chapter*{Úvod}
\addcontentsline{toc}{chapter}{Úvod} % přidá položku úvod do obsahu
V této práci se zaměřuji na návrh a výrobu laserového projektoru, který bude za pomoci páru zrcátek připevněných na galvanometrech rychle měnit směr laserového paprsku a tím vykreslovat obraz na promítací plochu.

Technologie rychle se pohybujícího laserového paprsku (laser scanning) je využívána v mnoha oblastech od jednoduchého promítání, efektů na diskotékách a Heads Up Displejů v letadlech či autech \cite{huds-in-driving}, přes čtení čárových kódů~\cite{history-of-barcode-scanning} a 3d tisk~\cite{laser-sintering} po 3D skenování modelů~\cite{3d-model-scan} i Zemského povrchu~\cite{heightmaps}.

Bohužel ale neexistují žádné uživatelsky přívětivé open-source platformy, kde by se s touto technologií mohli seznámit zájemci o její rozvíjení.

V této práci jsem se proto rozhodl pro tuto technologii vytvořit jednoduché uživatelské prostředí, ve kterém si i začínající kutil může vyzkoušet jak funguje.