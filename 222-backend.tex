\section{Lasershow}
\subsection{Úvod}

Program lasershow je~psaný v~jazyce c++. Tento jazyk je~kompilovaný a~díky tomu je~rychlý a~běží samostatně.
Kdybych využil interpretovaný jazyk, přidal bych k~operačnímu systému další vrstvu neovlivnitelných záseků -- interpreter~\cite{wiki-interpret}.

\fxnote{TODO: co~dela - co~do projektu dava ig}
Lasershow obsluhuje samotné vykreslování. Dostane-li od~UI programů 

Program byl inspirován projektem \url{https://github.com/tteskac/rpi-lasershow}\footnote{staženo 28.~12.~2023}.
Velkou výhodou oproti zmíněnému programu je~možnost s~programem interagovat i~v~průběhu promítání. Díky tomu je~možné měnit nastavení promítání bez nutnosti zastavit program a~ztratit všechny jeho data.

\subsection{Využité knihovny}
\subsubsection{ZeroMQ}
\subsubsection{ADCACDASFASDFAH sassyyy}
, toto rozhraní využívám pomocí knihovny ze~serveru \url{https://github.com}\footnote{\url{https://github.com/abelectronicsuk/ABElectronics_CPP_Libraries/tree/master/ADCDACPi}; staženo 2.~1.~2024}
\subsubsection{pigpio}
pwm

\subsection{Chování programu}
Tento program pomocí knihovny ZeroMQ zaregistruje vstupní TCP socket na~portu~5557 a~přihlásí se~na něm k~odběru zpráv, které do~něj publikují ostatní programy. Zárověn podobně zaregistruje výstupní socket na~portu~5556, do~kterého později bude posílat zprávy pro programy, které interagují s~uživatelem.

\fxnote{TODO: priklad zmq}\

\fxnote{ use \url{https://cs.overleaf.com/learn/latex/Code_Highlighting_with_minted} instead - function highlights}

\lstinputlisting[language=c++, style=code]{code_examples/zmq_server.cpp}
\lstinputlisting[language=c++, style=code]{code_examples/zmq_client.cpp}

Následně se~pomocí připojí k~D/A převodníku skrz rozhraní SPI\cite{wiki-SPI} a~čeká na~zprávy od~ostatních programů. Jakmile zprávu obdrží, zpracuje ji.
Pokud je~požadována změna nastavení, okamžitě ji~provede, aktuální nastavení si~uloží do~souboru lasershow.cfg a~přepočítá informace o~promítaném obrazu, je-li to~nutné.
(Například, změní-li se~nastavení trapezoid-horizontal, přepočítá souřadnice vykreslovaných bodů.) Jestliže je~požadováno vykreslení obrazu ze~souboru, začne obraz vykreslovat.
Při tom průběžně posílá informace o~stavu vykreslování do~výstupního socketu. I~při vykreslování obrazu tento program zpracovává zprávy a~pokyny ze~vstupního socketu.

\fxnote{TODO: co~to vlastne umi - umoznuje zive skrz UI~menit nastaveni - laseru, trapezoid souradnic, v~UI je~viditelny stav vykreslovani}

\fxnote{TODO: diagram programu}

\subsection{lepsi, nez rpi-lasershow}
benchmark ctvereckovanyho papiru je vlastne nanic, cuz ten kod je stejne rychlej, moje vyhoda je v tom, ze nemusim nacitat mezi snimkama, tudiz benchmark je spis kolik snimku za sekundu, ne: jak dlouho trva jeden snimek/kolik zvladnu vykreslit na jednom snimku
->jde mi o cas straveny mezi snimky

\section{wifi\_manager}

V rámci této práce byl vyvinut ještě jeden program, který se~přímo nepodílí ani na~projekci, ani na~interakci s~uživatelem.

Program wifi\_manager je~také napsaný v~jazyce JavaScript s~využitím runtime Node.js. Registruje se~ke stejným socketům jako lasershow, přijímá příkazy týkající se~nastavení WiFi na~Raspberry Pi~TCP socketem na~portu 5557 a~odesílá zpětnou vazbu na~TCP socket s~portem 5556.

\fxnote{TODO: jak se~komunikace s~lasershow odlisuje od~wifi\_managera}

\fxnote{TODO: ukazka(jak interaguje se~systemem, aby toho vseho dosahl lol)}

Hlavním úkolem tohoto programu je~správa a~konfigurace WiFi připojení na~Raspberry Pi. Přijímá příkazy od~ostatních programů a~nastavuje WiFi parametry na~základě těchto příkazů. Tím umožňuje uživatelům snadno a~pohodlně nastavit WiFi připojení na~svém zařízení.

Stejně jako lasershow, wifi\_manager také posílá zpětnou vazbu ostatním programům, aby informoval o~stavu a~změnách v~nastavení WiFi. Tímto způsobem je~zajištěna komunikace a~synchronizace mezi všemi programy v~laserovém projektoru.

Celkově wifi\_manager přispívá k~plynulému a~efektivnímu provozu laserového projektoru tím, že umožňuje snadnou správu a~konfiguraci WiFi připojení na~Raspberry Pi.