\chapter{Diskuze}

Existuje nemalé množství open-source implementací laserových projektorů.

Většina z~nich ale~není uživatelsky přívětivá, to~neznamená, že jsou tyto projekty špatné, ale~hodí se~uživatelům, kteří jsou pokročilejší programátoři. 
Jednou z~nich byl~i tento projekt inspirován.
Ta se~jmenuje rpi-lasershow a~je~dostupná z~\cite{rpi-lasershow}. Obsahuje základní vykreslovací program, který umí číst soubory formátu .ild. Její hlavní problém spočívá v~neexistujícím uživatelském prostředí.

Nad ostatní implementace vystupuje projekt openlase~\cite{openlase}. Ten~je~koncipován jako knihovna, kterou zájemci mohou začlenit do~vlastních projektů.
Hardware k~němu navžený ale~vyžaduje připojení k~počítači i~zásuvce, což by~mohlo být nepraktické, kdyby se~hodilo projekci předvést potenciálním zájemcům o~technologii naživo.

Grafické uživatelské prostředí obsahuje projektor vypracovaný v~diplomové práci Bc. Pavla Svobody~\cite{vut-chabr}. Ten~je~ale, stejně jako openlase, konstruován s~připojením k~počítači a~zásuvce.

Za zmínku také stojí projekt zpracovaný v~youtube videu kanálu \uv{Ben Makes Everything}~\cite{harddrive-projector-youtube}, který obsahuje i~aplikaci na~mobilní telefon. Ta ~s~projektorem komunikuje přes bluetooth. Tento projektor ale~uživatelům umožňuje vypsat jen~text, žádné vlastní obrázky ani~videa.
Zajímavé na~tomto projektu je, že je~založen na~hranolovém skeneru, který tvůrce vytvořil ze~starého hard-disku. Je~tedy poměrně dostupný pro~kohokoliv s~přístupem ke~3d tiskárně, která byla při výrobě projektoru použita.
Představení hranolových skenerů je~užitečné vzhledem k~jejich častému využití v~aplikacích jiných, než je~vykreslování obrazu. Grafiky vykreselné galvanometrovými skenery můžou být zajímavější, než ty~vykreslené skenery hranolovými.

Když opustíme sféru open-source, samozřejmě existuje spousta komerčních řešen. Například od~firem Pangolin nebo Laserworld. Ty~jsou často výkonnější, než výše popsané projekty a~jsou využívané pro~vytváření profesionálních laserových efektů.

Velice zajímavým zařízením je~Laser Cube~\cite{lasercube} od~firmy Wicked Lasers. Je ~to~komerční laserový projektor pro~širokou veřejnost s~velice jednoduchým uživatelským prostředím. Open-source řešení včetně této práce by ~se~k němu daly přirovnat jako alternativy pro~technicky zdatné jedince, kteří jsou schopni s~technologií pracovat sami.
