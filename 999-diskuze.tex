\fxnote{TODO udelals to~vubec dobre? porovnej se~s ostatnima}
\chapter{Diskuze}
\section{ruzne technologie}
\cite{scanning-handbook} str. 394 \\
Digital micromirror devices (DMDs) and
liquid crystal displays (LCDs) have also captured the field of~image projection away from
oscillating scanners.
uz existuje MEMS \cite{mems-review}- pcb, co~ma~X~i Y~v jednom zrcatku, je~super a~vsechno jiny prave nahrazuje, na~user levelu je~i docela podobny galvum - jedna civka x~druha y~- overit,, idk, jesti tam civky nejsou proste pod tim a~nejenom v~kloubech

\fxnote{vytvoril jsem projektor, ten je~mozne sestavit s~minimální znalostí elektroniky, staci umet pajet, jde mi~o to~zajemcum predstavit jak to~programovat, jak funguji vektorove formaty, ne~jak funguje hw, ten se~totiz meni - MEMS}

\fxnote{kam pokracovat: api pro random programy /just na~github napsat lol pouzivejte tyhle funkce na~ovladani DAC ahah - aby si~mohli pokrocilejsi kutilove taky hrat a~randomly posouvat laser ig}

\section{další zpracování tématu}
udelal jsem to~dobre? vybral jsem si~dobry technky?
like byl by~lepsi ten harddrive z~yt?
nebo fakt to~melo byt napajeny z~baterek a~ne~ze~zasuvky?

ze hej ze~\href{https://dspace.vutbr.cz/bitstream/handle/11012/38621/final-thesis.pdf?sequence=-1}{typek z~vut} udelal kinda kurva podobnej HW~jak ja, ale ja~to~mam trochu jinak, cuz jsem o~tom nevedel, ale ofc moje je~lepsi :))
also to~delala hromada dalsich lidi na~internetu ten hw, also od~gh.com/tteskac mam executable, kterou jsem ale totalne ze~rozsiril a~taky jsem pridal vsechno moje genialni ui~muhahahah

\B{ze este dalsi zpracovani: (19.10.2023 vsechny dostupne)}
\begin{enumerate}
  \item used/modified code
        \begin{itemize}
          \item \url{https://github.com/marcan/openlase/blob/master/tools/svg2ild.py}
          \item \url{https://github.com/tteskac/rpi-lasershow}
          \item \url{https://github.com/sabhiram/raspberry-wifi-conf/blob/master/app/wifi\_manager.js}
          \item \url{http://www.electronicayciencia.com/wPi_soft_lcd/}
          \item \href{https://dspace.vutbr.cz/bitstream/handle/11012/38621/final-thesis.pdf?sequence=-1}{typek z~vut}
        \end{itemize}
  \item dalsi zpracovani stejny projekty
        \begin{itemize}
          \item \url{https://www.instructables.com/Arduino-Laser-Show-With-Real-Galvos/}
          \item \url{https://github.com/tteskac/rpi-lasershow}
          \item \url{https://www.instructables.com/DIY-STEPDIR-LASER-GALVO-CONTROLLER/}
          \item \href{https://youtu.be/u9TpJ-_hBR8?si=mHy-UrptZZJ0Xu5-}{borec na~yt~hard-drive text gut}
        \end{itemize}
  \item other useful thingies
        \begin{itemize}
          \item \url{https://hackaday.io/project/172284-galvo-laser-cutterengraver}

          \item \url{https://hackaday.io/project/172284/instructions}

          \item \url{https://learn.adafruit.com/mcp4725-12-bit-dac-with-raspberry-pi/hooking-it-up}
          \item \url{https://www.ilda.com/resources/StandardsDocs/ILDA_IDTF14_rev011.pdf}
          \item cool demos \url{https://marcan.st/projects/openlase/}
          \item \url{https://www.youtube.com/watch?v=u9TpJ-_hBR8}
        \end{itemize}
  \item read
        \begin{itemize}
          \item \url{https://www.laserworld.com/en/glossary-definitions/90-t/2797-ttl-modulation-en.html}
        \end{itemize}

\end{enumerate}
