% !TeX root = text.tex
\chapter{Software}

Tato kapitola se~zabývá softwarovou výbavou laserového projektoru. Popisuje klíčové programy, jejich funkce a~způsob, jakým mezi sebou komunikují.
Mezi tyto programy patří program lasershow, který obsluhuje vykreslování, programy pro~interakci s~uživatelem (dále \uv{frontendové programy}) jako UI, web\_ui a~discord\_bot, a~také program wifi\_manager, který spravuje wifi připojení Raspberry Pi.
Programy wifi\_manager a~lasershow se~dají označit za~backendové programy, protože neinteragují s~uživatelem.

V neposlední řadě popisuje také instalační skript, který výrazně zjednodušuje instalaci všech zmíněných programů a~jejich závislostí.

Program lasershow umí číst projekční soubory ve~formátu .ild.
Uživatel může takový soubor vytvořit například v~softwaru Laserworld Showeditor V7~\cite{showeditor} nebo může do~frontendových programů nahrát soubor v~populárním vektorovém formátu .svg a~frontendové programy využijí skript svg2ild.py dostupný~z~\cite{svg2ild}.

% O~řízení galvanometrů se~stará program lasershow, který je~psaný v~jazyce C++ pro~maximální rychlost. Tento program běží na~pozadí a~čeká na~příkazy od~programů určených k~interakci s~uživatelem. Na~tento program se~zaměřuje kapitola lasershow. \fxnote{TODO: odkaz}

% Dále jsou tu~programy, které se~starají o~interakci s~uživatelem. Tyto programy přijímají příkazy od~uživatele a~posílají je~programu lasershow. Navíc od~lasershow získávají výstup, který následně zprostředkovávají uživateli; důkladněji popsáno v~kapitole~\ref{sec:comms}.
% Mezi tyto programy patří programy UI, web\_ui a~discord\_bot. Program UI~spravuje OLED displej, přijímá od~uživatele vstup rotačním enkodérem a~je~psaný v~C++ pro~jednodušší interakci s~hardwarem. Program web\_ui využívá runtime Node.js, ve~kterém je~nenáročné vytvořit http server dostupný z~lokální sítě. \fxnote{(A)?} Program discord\_bot, také využívající Node.js, přijímá příkazy z~chatovací aplikace discord a~je~přístupný i~přes internet.

% Nakonec je~tu~program wifi\_manager, ten~spravuje wifi připojení RPi, je~psaný v~Node.js a~komunikuje s~programy, které interagují s~uživatelem stejně jako program lasershow.

\section{komunikace mezi programy} \label{sec:comms}
Všechny tyto programy jsou propojeny síťovými sockety zprostředkovanými knihovnou ZeroMQ, která nabízí frontu\footnote{Ve frontě jsou zprávy seřazeny od~té nejdříve odeslané.} zpráv, bez potřeby samostatně běžícího brokeru.

Tato knihovna je~využita k~vytvoření dvou socketů, jedním lasershow přijímá příkazy od~uživatele prostřednictvím ostatních programů (vstupní socket na~portu 5557, viz obr. \ref{fig:tcp5557}) a~do druhého posílá informace ostatním programům (výstupní socket na~portu 5556, viz obr. \ref{fig:tcp5556}), aby je~zprostředkovaly uživateli. Do~prvního zmíněného posílají progamy interagující s~uživatelem příkazy pro programy lasershow a~wifi\_manager. Do~druhého posílá lasershow informace o~stavu a~změnách nastavení  a~také wifi\_manager informace o~stavu a~změnách v~nastavení WiFi.

Příkazy pro programy lasershow a~wifi\_manager vypadají následovně
\fxnote{TODO: příklady příkazů pro lasershow a~wifi\_manager z~\url{https://github.com/phuid/laser_projector/blob/master/README.md}}
\fxnote{TODO: příklady status infos od~lasershow a~wifi\_manager z~\url{https://github.com/phuid/laser_projector/blob/master/README.md}}

\begin{figure}[!htb]
  \centering
  \includegraphics[width=0.5\textwidth]{img/tcp5557.png}
  \caption{\label{fig:tcp5557}komunikace mezi programy vstupním socketem na~portu 5557}
\end{figure}
\begin{figure}[!htb]
  \centering
  \includegraphics[width=0.5\textwidth]{img/tcp5556.png}
  \caption{\label{fig:tcp5556}komunikace mezi programy výstupním socketem na~portu 5556}
\end{figure}

\section{lasershow}

Program lasershow je psaný v jazyce c++, který je kompilovaný a obecně považovaný za jeden z nejrychlejších jazyků. Druhé zmíněné se hodí, jelikož chceme vykreslovat co možná nejrychleji.

Tento program zaregistruje vstupní TCP socket na portu 5557 a knihovnou ZeroMQ se na něm přihlásí k odběru zpráv, které do něj publikují ostatní programy. Zárověn podobně zaregistruje výstupní socket na portu 5556, do kterého později bude posílat zprávy pro programy, které interagují s uživatelem.

Následně se připojí k DAC a čeká na zprávy od ostatních programů. Jakmile zprávu obdrží, zpracuje ji a pokud je požadována změna nastavení, okamžitě ji provede a aktuální nastavení si uloží do souboru, jestliže je požadováno vykreslení obrazu ze souboru, začne obraz vykreslovat. Při tom průběžně posílá informace o stavu vykreslování do výstupního socketu. I při vykreslování obrazu tento program zpracovává zprávy a pokyny ze vstupního socketu.

Program byl původně převzat z projektu \url{https://github.com/tteskac/rpi-lasershow}\footnote{staženo 28.~12.~2023}, následně byl ale přepsán skoro ve všech ohledech a z původního programu zbylo asi 20 řádků.
\fxnote{TODO: odkud jsem to vzal a prepsal a jak moc jsem toho udelal a s jakymy vysledky}

\fxnote{TODO: diagram programu}

\fxnote{TODO: priklad zmq}\

\lstinputlisting[language=c++, style=code]{code_examples/zmq_server.cpp}
\lstinputlisting[language=c++, style=code]{code_examples/zmq_client.cpp}


\section{wifi\_manager}

V rámci této práce byl vyvinut ještě jeden program, který se přímo nepodílí ani na projekci, ani na interakci s uživatelem.

Program wifi\_manager je také napsaný v jazyce JavaScript s využitím runtime Node.js. Registruje se ke stejným socketům jako lasershow, přijímá příkazy týkající se nastavení WiFi na Raspberry Pi TCP socketem na portu 5557 a odesílá zpětnou vazbu na TCP socket s portem 5556.

\fxnote{TODO: jak se komunikace s lasershow odlisuje od wifi\_managera}

\fxnote{TODO: ukazka(idk what)}

Hlavním úkolem tohoto programu je správa a konfigurace WiFi připojení na Raspberry Pi. Přijímá příkazy od ostatních programů a nastavuje WiFi parametry na základě těchto příkazů. Tím umožňuje uživatelům snadno a pohodlně nastavit WiFi připojení na svém zařízení.

Stejně jako lasershow, wifi\_manager také posílá zpětnou vazbu ostatním programům, aby informoval o stavu a změnách v nastavení WiFi. Tímto způsobem je zajištěna komunikace a synchronizace mezi všemi programy v laserovém projektoru.

Celkově wifi\_manager přispívá k plynulému a efektivnímu provozu laserového projektoru tím, že umožňuje snadnou správu a konfiguraci WiFi připojení na Raspberry Pi.

\section{UI}

Program UI~je také psaný v~jazyce c++ a~využívá knihovnu WiringPi, která umožňuje jednoduchou komunikaci s~GPIO piny Raspberry Pi. Tento program ovládá OLED displej, který je~připojený na~Raspberry Pi~pomocí rozhraní I2C, a~přijímá vstup od~uživatele čtením rotačního enkodéru s~tlačítkem.

Program se~při začátku exekuce pomocí knihovny ZeroMQ přihlásí ke~vstupnímu socketu a~k odběru zpráv z~výstupního TCP socketu, kam publikuje zprávy o~stavu vykreslování program lasershow. Dále si~pomocí knihovny wiringPi zaregistruje zpracovávání přerušení z~enkodéru a~tlačítka na~něm a~čeká buď na~interakci s~uživatelem, který by~skrz něj poslal zprávy programu lasershow, nebo na~zprávy od~lasershow, které by~zobrazil uživateli.

\fxnote{TODO: diagram programu}

\section{web\_ui}

Narozdíl od~předchozích dvou zmiňovaných programů je~program web\_ui psaný v~jazyce javascript, ten nepatří mezi nejrychlejší, ale díky runtime Node.js a~knihovnám http a~formidable v~něm bylo časově nenáročné vytořit http web server.

Tento server běží na~portu 3000 a~je dostupný z~lokální sítě (tzn. přímo z\~Raspberry Pi~na adrese http://localhost:3000 nebo z~jakéhokoliv zařízení na~stejné lokální síti na~ip adrese RPi).
Program je~využíván pro jednoduchou interakci s~uživatelem, který může pomocí webového prohlížeče ovládat laserový projektor pár kliknutími i~zadávat vlastní příkazy klávesnicí.

\fxnote{na webu jsou konzole pro ssh, wifiman a~lasershow, taky fast project forms}

\fxnote{TODO: příklad http serveru}
\lstinputlisting[language=JavaScript, style=code]{code_examples/http_static_files.js}

Stejně jako program UI~za pomoci knihovny ZeroMQ tento program odebírá z~výstupního socketu zprávy o~průběhu vykreslování od~programu lasershow a~odesílá mu~pokyny uživatele na~vstupní socket.

\fxnote{TODO: příklad přihlášení k~socketům v~js}

\fxnote{TODO: xterm + ssh}

\section{discord bot}

Posledním programem, který je~využíván k~interakci s~uživatelem je~discord\_bot, který je~také psaný v~jazyce javascript v~runtime Node.js, stejně jako předchozí programy se~přihlásí k~socketům knihovnou zmq, ale na~rozdíl od~nich tento program může interagovat s~uživatelem přes internet ať už je~kdekoliv na~světě.
Pomocí knihovny discord.js se~přihlásí k~předem vytvořenému bot účtu, který může na~předem vytvořeném discord serveru čekat na~zprávy od~uživatele, ty~posílat do~vstupního socketu a~posílat uživateli zpětnou vazbu, kterou příjme z~výstupního socketu.



\section{Instalační skript}
Nedílnou součástí softwarové výbavy projektoru je~instalační skript.
Ten je~psaný v~příkazovém jazyce bash, který odpovídá sekvenci příkazů v~příkazovém řádku.
Instalační skript umožňuje instalaci celého projektu pouze třemi příkazy, viz~ukázka kódu~\ref{list:installcmds}.

\begin{code}
  \captionof{listing}{\label{list:installcmds} Příkazy potřebné k~instalaci projektu}
\begin{minted}[frame=lines,fontsize=\footnotesize,linenos]{shell}
  git clone https://github.com/phuid/laser_projector.git
  cd laser_projector
  bash install.sh
\end{minted}
\end{code}

Instalační skript stáhne a~nainstaluje veškeré závislosti a~knihovny ostatních programů. Stáhne také samotný interpreter Node.js.
Následně zkompiluje programy psané v~C++ a~nainstaluje a~nastaví služby potřebné k~vysílání hotspotu. Poté pomocí manažeru procesů pm2 dostupného z~\cite{pm2} nastaví automatické zapínání ostatních programů a~po~potvrzení uživatelem systém restartuje, aby~provedené změny nabyly efekt.
