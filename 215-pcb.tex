\section{HAT}
Pro ovládání výše popsaného hardwaru je~zapotřebí několik specifických obvodů.
Kvůli jejich specifičnosti tyto obvody nejsou volně dostupné k~zakoupení na~předem vytvořených destičkách. Proto bylo zapotřebí je~z jednotlivých součástek vyrobit na~míru.

Obvody byly navrženy v~programu KiCad...\fxnote{bud spojit vety, nebo k~prvni neco jeste dopsat}
Následně pro ně v~tomtéž programu byla nadesignována deska plošných spojů. Na~této desce se~vyskytují obvody \fxnote{todo dac+amps, bat\_probe, -15V}.
Kromě nich byly na~desku přidány konektory k~jednotlivým barevným vstupům laseru, LCD displeji a~k rotačnímu enkodéru, které jsou přímo napojeny na~40 pinový GPIO konektor Raspberry Pi.
Deska byla designována jako tzv. HAT, to~znamená, že sama na~tomto konektoru drží a~nezabírá o~moc víc místa, než samotné Raspberry Pi.
\fxnote{TODO: obrazek desky (maybe mounted)}

\subsection{Zdroj $-15$~V}

\subsection{obvod pro generování analogového signálu}
Jak popsáno v~sekci \ref{sec:my-galvos}, řídící deska galvanometrů přijmá dva bipolární diferenciální analogové signály v~rozpětí $-5$~V až $+5$~V.

Obvod, který se~stará o~vytváření tohoto signálu je~založený na~obvodu ze~zdroje~\cite{lasershow-with-real-galvos}.
Vytváření tohoto signálu je~rozděleno do~dvou částí. Nejdříve DAC (digital-to-analog converter, D/A převodník) připojený k~RPi vytvoří signál v~rozpětí 0 až 5~V a~následně je~tento signál pomocí operačního zesilovače převeden na~požadované rozpětí, tj. $-15$~V až $+15$~V.
Jednotlivé části tohoto obvodu jsou blíže popsány v~následujících kapitolách. Celé zapojení je~vidět na~obrázku \ref{fig:dac_board}.
\fxnote{unreadable text, make schem more compact}
\begin{figure}[!htb]
  \centering
  \includegraphics[width=1\textwidth]{img/dac_board.png} 
  \caption{\label{fig:dac_board}Zapojení DAC a~zesilovačů k~RPi a~řídící desce galvanometrů}
\end{figure}

\subsubsection{dac\cite{mcp4822-dsh}}
K generování signálu v~rozpětí 0--5~V byl využit dvoukanálový D/A převodník\footnote{obvod, který na~základě instrukcí přijatých digitálně generuje analogové napětí} MCP4822.
Tento čip podporuje komunikaci přes rozhraní SPI, pracuje s~napájecím napětím 5~V a~s 12bitovým rozlišením (je schopen vygenerovat 4~096 různých napětí) na~dvou kanálech.

RPi komunikuje s~čipem pomocí rozhraním SPI.
\fxnote{TODO more spec}
Tato knihovna poskytuje následující funkce, se~kterými pracuji v~mém kódu.
\begin{itemize}
\item
\lstinline[language=C]!bool mcp4822_initialize();!
\item
\lstinline[language=C]!bool mcp4822_set_voltage(mcp4822_channel_t channel, uint16_t value_mV);!
\item
\lstinline[language=C]!void mcp4822_deinitialize();!
\end{itemize}
\subsubsection{amps\cite{tl082-dsh}}
K modifikaci signálu z~DAC na~bipolární diferenciální analogový signál slouží pro každý kanál jeden čip TL082, který obsahují dva operační zesilovače. Ty~jsou zapojeny dle schématu na~obrázku \ref{fig:ilda_amps-scheme}.

Signál první operační zesilovač zesílí a~posune dle nastavení potenciometrů Ygain(zesílení) a~Yoffset(posun) a~zároveň invertuje. Tento invertovaný signál následně druhý operační zesilovač opět invertuje, získav základní signál pro řídící desku galvanometrů.

\begin{figure}[!htb]
  \centering
  \includegraphics[width=1\textwidth]{img/ilda_amps.png} 
  \caption{\label{fig:ilda_amps-scheme} Zapojení čipu TL082 pro jeden kanál řídící desky galvanometrů}
\end{figure}

\fxnote{TODO more spec}
Tyto čipy mi~napěťové rozpětí zvýší z~0--5~V na~$-15$~V až $+15$~V.

zesilovac - cteni baterek \url{https://is.muni.cz/el/sci/jaro2017/F5090/um/E17_P8.pdf}
