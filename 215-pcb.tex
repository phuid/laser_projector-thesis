\section{HAT deska plošných spojů}
Pro ovládání výše popsaného hardwaru je~zapotřebí několik specifických obvodů.
Kvůli jejich specifičnosti tyto obvody nejsou volně dostupné k~zakoupení na~předem vytvořených destičkách. Proto bylo zapotřebí je~z~jednotlivých součástek vyrobit na~míru.

Obvody byly navrženy v~programu KiCad, celé schéma desky je k nalezení mezi přílohami s označením~\ref{fig:pcb-schematic-full}. Následně pro ně v~tomtéž programu byla nadesignována deska plošných spojů. Mezi obvody patří:
\begin{itemize}
  \item Zdroj $-15$~V pro galvanometry.
  \item Generátor signálu pro řídící desku galvanometrů.
  \item Voltmetr baterií.
\end{itemize}
\fxnote{TODO: fotka desky}
\fxnote{TODO: odkaz na kicad\_sch a kicad\_pcb soubory v priloze}

Kromě nich byly na~desku přidány konektory k~jednotlivým barevným vstupům laseru, LCD displeji a~k rotačnímu enkodéru, které jsou přímo napojeny na~40 pinový GPIO konektor Raspberry Pi.
Deska byla designována jako tzv. HAT, to~znamená, že sama na~tomto konektoru drží a~nezabírá o~moc víc místa, než samotné Raspberry Pi.
\fxnote{TODO: obrazek desky (maybe mounted) (deska ještě nedošla)}

\subsection{Zdroj $-15$~V}\label{sec:negative-ps}
Napětí $-15$~V je získáváno obvodem napěťového invertoru založeném na obvodu ze zdroje~\cite{ampalyzer}. Jeho zapojení je na obrázku~\ref{fig:negative-ps}, potřebuje zdroj napětí 15V.

V aktuálním stavu tento zdroj nefunguje, to bude brzo napraveno.

\begin{figure}[htb]
  \centering
  \includegraphics[width=0.9\textwidth]{img/negative-ps.jpg}
  \caption{\label{fig:negative-ps} Zapojení invertujícího obvodu}
\end{figure}

Centrem obvodu je integrovaný obvod AOZ1282 od výrobce Alpha \& Omega Semiconductor označený U6. Tento integrovaný obvod obsahuje spínací transistor (ve zjednodušeném schématu prvek SW), PWM regulační obvod pracující na frekvenci 450 kHz s napěťovou referencí 0,8 V, který reguluje čas připojení induktoru L1.
K němu je připojen bootstrapový kondenzátor C1, ten zajišťuje plovoucí buzení pro integrovaný spínač.
Dále je k němu připojený výkonový induktor L1, jehož hodnota byla zvolena dle rovnic uvedených ve zdroji~\cite{basic-calc-boost}, respektive podle dále uvedené rovnice~\ref{equ:inductor-calc}.~\cite{ampalyzer}

\begin{equ}[H]
  \centering
  \begin{math}
    L = \frac{-U_{OUT}\times U_{IN}}{0,4 \times 2 \times I_{OUT} \times f_{s} \times \left ( U_{IN} - U_{OUT} \right )} = \frac{- \left (-15 V \right )\times 15 V}{0,4 \times 2 \times 1 A \times 450 kHz \times \left ( 15 V - \left (-15 V \right ) \right )} \approx 21 \mu H
  \end{math}
  \caption{\label{equ:inductor-calc} Výpočet ideální indukčnosti cívky pro invertující obvod}
\end{equ}

\It{
  L --- Indukčnost spínaného induktoru \\
  U$_{IN}$ --- Vstupní napětí do invertujícího obvodu \\
  U$_{OUT}$ --- Výstupní napětí z invertujícího obvodu \\
  I$_{OUT}$ --- Výstupní proud z invertujícího obvodu \\
  f$_{s}$ --- Frekvence spínacího regulátoru \\
}

Na FB pin integrovaného obvodu je připojen napěťový dělič tvořený odpory R13 a R14, který integrovanému obvodu dodává zpětnou vazbu o výstupním napětí.
Hodnoty R13 a R14 jsou voleny tak, aby při 15 V, tedy požadovaném výstupním napětí, bylo na výstupu děliče napětí 0,8 V, tedy referenční napětí integrovaného spínaného regulátoru.
Schottkyho dioda SS56, označená D1, slouží k zadržení změny polarity induktoru. Usměrňovací dioda D2 je v propustném stavu při prvotním spuštění měniče, kdy je přes ní napájen U1 po dobu náběhu výstupního záporného napětí.
Posledním prvkem je výstupní vyhlazovací filtr tvořený kondenzátory C3 a C4 společně s induktorem L2.
Jeho úkol je minimalizovat výstupní napěťového zvlnění zdroje.~\cite{ampalyzer}

\subsection{Generátor analogového signálu}\label{sec:ilda-signal-gen}
Jak je popsáno v~sekci~\ref{sec:my-galvos}, řídící deska galvanometrů přijmá dva bipolární diferenciální analogové signály v~rozpětí $-5$~V až $+5$~V.

Obvod, který se~stará o~vytváření tohoto signálu, je~založený na~obvodu ze~zdroje~\cite{lasershow-with-real-galvos}.
Vytváření tohoto signálu je~rozděleno do~dvou částí. Nejdříve D/A~převodník připojený k~RPi vytvoří signál v~rozpětí 0 až 5~V a~následně je~tento signál pomocí invertujících operačních zesilovačů převeden na~požadované rozpětí a invertován.
Jednotlivé části tohoto obvodu jsou blíže popsány v~následujících kapitolách.

\subsubsection{D/A převodník}
K generování signálu v~rozpětí 0--5~V byl využit dvoukanálový D/A převodník\footnote{D/A převodník je obvod, který na~základě instrukcí přijatých digitálně generuje analogové napětí.} MCP4822.
Tento čip podporuje komunikaci přes rozhraní SPI, pracuje s~napájecím napětím 5~V a~s 12bitovým rozlišením (je schopen vygenerovat 4~096 různých napětí) na~dvou kanálech~\cite{mcp4822-dsh}.
RPi komunikuje s~čipem pomocí rozhraní SPI popsaném v kapitole~\ref{sec:spi}.

\subsubsection{Operační zesilovače~\cite{tl082-dsh}}
K modifikaci signálu z~DAC na~bipolární diferenciální analogový signál slouží pro každý kanál jeden čip TL082, který obsahuje dva operační zesilovače. Ty~jsou zapojeny dle schématu na~obrázku~\ref{fig:ilda_amps-scheme}.

Signál první operační zesilovač zesílí/zeslabí a~vertikálně posune dle nastavení potenciometrů Ygain (zesílení) a~Yoffset (posun) a~zároveň invertuje. Tento invertovaný signál následně druhý operační zesilovač opět invertuje. Tím získavá základní signál pro řídící desku galvanometrů.

\begin{figure}[htb]
  \centering
  \includegraphics[width=1\textwidth]{img/ilda_amps.png}
  \caption{\label{fig:ilda_amps-scheme} Zapojení čipu TL082 pro jeden kanál řídící desky galvanometrů}
\end{figure}

\subsection{Voltmetr baterií}
Obvod voltmetru baterií je vidět na obrázku \ref{fig:bat_probe}.

\begin{figure}[htb]
  \centering
  \includegraphics[width=0.6\textwidth]{img/bat_probe.jpg}
  \caption{\label{fig:bat_probe} Schéma obvodu voltmetru baterie}
\end{figure}

Jako voltmetr baterií slouží A/D převodník ADC0831 od firmy Texas Instruments Incorporated~\cite{adc0831-dsh}. Ten je označen U5 a je zapojen společně s operačním zesilovačem LM358 od stejného výrobce, který je označen U4.
Operační zesilovač je zapojen jako rozdílový zesilovač podle zdroje~\cite{odcitacka} tak, aby od napětí baterií, které se může pohybovat v rozsahu 6~V až 8,4~V, odečítal 5~V.
Díky tomu se napětí, která měří A/D převodník pohybují mezi hodnotami 1~V a 3,4~V. S jeho osmibitovým rozlišením a referenčním napětím 5~V v tomto rozpětí může naměřit 122 různých napětí.

A/D převodník tyto data posílá do Raspberry Pi pomocí rozhraní SPI a to s nimi dále pracuje.
